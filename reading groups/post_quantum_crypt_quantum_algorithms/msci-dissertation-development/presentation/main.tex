\documentclass[10pt]{beamer}

\usetheme[progressbar=frametitle]{metropolis}
\usepackage{appendixnumberbeamer}

\usepackage{booktabs}
\usepackage[scale=2]{ccicons}

\usepackage{pgfplots}
\usepgfplotslibrary{dateplot}

\usepackage{xspace}
\newcommand{\themename}{\textbf{\textsc{metropolis}}\xspace}


\setbeamercolor{alerted text}{fg=teal}
\setbeamercolor{emph}{fg=teal}
\renewcommand<>{\emph}[1]{%
  {\usebeamercolor[fg]{emph}\only#2{\itshape}#1}%
}

\title{A whirlwind introduction \newline to bounded linear operators on Hilbert spaces}
\subtitle{MATH491: Mathematics Dissertation}
% \date{\today}
\date{}
\author{Luke Mader}
\institute{Lancaster University}

\usepackage[utf8]{inputenc}\usepackage[a4paper,
                                      %right = 3cm
                                      lines = 42
                                      ]{geometry}
\setlength{\textwidth}{15cm}
\usepackage[export]{adjustbox}
\usepackage{amsmath,
            amssymb,
            gensymb,
            fancyhdr,
            esint,
            tcolorbox,
            epigraph,
            mathtools,
            titlesec,
            etoolbox,
            lastpage,
            scrextend,
            framed,
            enumitem,
            lipsum}
\usepackage[colorlinks=true,
            linkcolor=black,
            anchorcolor=black,
            citecolor=black,
            filecolor=black,
            menucolor=black,
            runcolor=black,
            urlcolor=black]{hyperref}

%\usepackage[nottoc,notlot,notlof]{tocbibind} % add bib to contents page
\usepackage[backend=biber,
            style=alphabetic,
           ]{biblatex}
\addbibresource{references.bib}


\usepackage{textcomp} % i forget this package's purpose

\setlength{\parindent}{0pt}

%headers
\usepackage{fancyhdr}
\usepackage{ifthen}

\pagestyle{fancy}
\fancyhead{}
\fancyhead[L]{\ifthenelse{\isodd{\value{page}}}{\leftmark}{\rightmark}}
\fancyhead[R]{\thepage}
\renewcommand{\headrulewidth}{0pt}
\fancyfoot{}


\usepackage{amsthm}
\theoremstyle{definition}
\newtheorem{theorem}{Theorem}
\numberwithin{theorem}{section}
\numberwithin{theorem}{subsection}
\newtheorem{definition}[theorem]{Definition}
\newtheorem{lemma}[theorem]{Lemma}
\newtheorem{corollary}[theorem]{Corollary}
\newtheorem{proposition}[theorem]{Proposition}
\newtheorem{example}[theorem]{Example}
\newtheorem{remark}[theorem]{Remark}
\newtheorem{axiom}{Axiom}

\newcommand{\set}[1]{\left\{#1\right\}}
\newcommand{\N}{\mathbb{N}}
\newcommand{\R}{\mathbb{R}}
\newcommand{\Z}{\mathbb{Z}}
\newcommand{\Q}{\mathbb{Q}}
\newcommand{\C}{\mathbb{C}}
\newcommand{\M}{\mathrm{M}}
\newcommand{\F}{\mathbb{F}}
\newcommand{\HS}{\mathcal{H}}
\newcommand{\KS}{\mathcal{K}}
\newcommand{\G}{\operatorname{G}}

\newcommand{\eps}{\varepsilon}
\newcommand{\id}{\text{id}}
\newcommand{\dist}[1]{\text{dist}\left(#1\right)}
\newcommand{\ip}[1]{\left\langle #1\right\rangle}
\newcommand{\conjugate}[1]{\mkern 1.5mu\overline{\mkern-1.5mu#1\mkern-1.5mu}\mkern 1.5mu}
\newcommand{\vect}[1]{\boldsymbol{#1}}


\DeclarePairedDelimiter\abs{\lvert}{\rvert}%
\DeclarePairedDelimiter\norm{\lVert}{\rVert}%

% Swap the definition of \abs* and \norm*, so that \abs
% and \norm resizes the size of the brackets, and the
% starred version does not.
\makeatletter
\let\oldabs\abs
\def\abs{\@ifstar{\oldabs}{\oldabs*}}
%
\let\oldnorm\norm
\def\norm{\@ifstar{\oldnorm}{\oldnorm*}}
\makeatother


\newcommand{\op}[1]{\norm{#1}_\text{op}}



% bra-ket notation
% uses package mathtools
\DeclarePairedDelimiter\bra{\langle}{\rvert}
\DeclarePairedDelimiter\ket{\lvert}{\rangle}
\DeclarePairedDelimiterX\braket[2]{\langle}{\rangle}{#1 \vert #2}
\DeclarePairedDelimiterX\ketbra[2]{\delimsize\vert}{\delimsize\vert}{#1 \rangle\langle #2}

\newcommand*{\rvec}[1]{\left( #1\right)}
\newcommand*{\tr}[1]{\operatorname{tr}\left(#1\right)}
\newcommand*{\range}[1]{\operatorname{range}\left(#1\right)}
\renewcommand*{\ker}[1]{\operatorname{ker}\left(#1\right)}
\newcommand{\dom}[1]{\operatorname{Dom}\left(#1\right)}


\begin{document}
\metroset{block=fill}
\maketitle

\begin{frame}{Why are we here?}
    \begin{enumerate}
        \item From vector spaces to Hilbert spaces.
        \vspace{1cm}
        \item Some types of linear operators.
        \vspace{1cm}
        \item The spectral theorem for compact self-adjoint linear operators.
    \end{enumerate}
\end{frame}

\begin{frame}{It all starts with vector spaces}
  Our fundamental building block at the foundation of everything is a {\emph{complex vector space}} $V$.

  \medskip

  We allow our vector spaces to be {\emph{infinite-dimensional}}: there exists no finite basis for $V$.

  \medskip

  MATH220 introduced the {\emph{inner product}}, \[\ip{\cdot, \cdot}: V \to \C.\]


  and the {\emph{norm induced by the inner product:}} \[\norm{\cdot} \coloneqq \sqrt{\ip{\cdot,\cdot}}\]

\end{frame}

% \begin{frame}{Norms}
%     \only<1, 2, 3>{
%     \begin{alertblock}{Norm}
%       A {\emph{norm}} on $V$ is a function $\norm{\cdot}: V \to \C$ such that for all $x, y \in V$ and $\lambda \in \C$,
%       \begin{enumerate}
%         \item $\norm{x + y} \leq \norm{x} + \norm{y}$,
%         \item $\norm{\lambda x} = \abs{\lambda}\norm{x}$,
%         \item $\norm{x} = 0$ if and only if $x = 0$
%       \end{enumerate}
%   \end{alertblock}
%     }
%     \only<2,3>{
%     It turns out that the function $\norm{\cdot} \coloneqq \sqrt{\ip{\cdot,\cdot}}$ is a norm, which we call the {\emph{norm induced by the inner product}}.
%
%     }
%     \only<3>{
%     It is the only type of norm that we'll be interested in!
%     }
%
% \end{frame}

\begin{frame}{The norm represents distance...}
    For all vectors $x$ and $y$, the value of \[d(x,y) \coloneqq \norm{x - y}\] can be thought of representing the distance between them!

    \bigskip

    We say that the {\emph{norm induces a metric}} on $V$.
\end{frame}

\begin{frame}{...so we can define the convergence of vectors}

    \begin{alertblock}{Convergence of vectors}
      For some vector space $V$ with a norm $\norm{\cdot}$, we say that a sequence of vectors $(x_n)$ in $V$ {\emph{converges}} to $x \in V$ if
      \begin{equation*}
          \norm{x_n - x} \to 0 \quad \text{as } n \to \infty.
      \end{equation*}
    \end{alertblock}
\end{frame}

\begin{frame}{Cauchy sequences in vector spaces}
    In $\R$ and $\C$, all Cauchy sequences converge.\\

    Is this true in any vector space? \\

    \pause {\begin{center}\Huge \textbf{No!} \end{center}}

    % \pause
    %
    % \begin{block}{}
    %   The vector space of {\emph{one variable complex polynomials}}, $\C[X]$, with the inner product
    %   \begin{equation*}
    %       \ip{f,g} \coloneqq \sum_{n=0}^\infty a_n \overline{b_n}
    %   \end{equation*}
    %   for any polynomials $f = \sum_{j=0}^\infty a_j X^j$ and $g = \sum_{k = 0}^\infty b_k X^k$, and where our norm is $\norm{\cdot} = \sqrt{\ip{\cdot, \cdot}}$, \newline \textbf{does not have that every Cauchy sequence converges.}
    % \end{block}
\end{frame}


\begin{frame}{Hilbert Spaces}
    This motivates the definition of a Hilbert space:

    \bigskip

    \begin{alertblock}{Hilbert Space}
        A {\emph{Hilbert space}} is a vector space $\HS$ equipped such that we have
        \begin{itemize}
            \item Some inner product,  $\ip{\cdot, \cdot}$,
            \item The norm induced by the inner product, $\norm{\cdot} = \sqrt{\ip{\cdot, \cdot}}$,
            \item Every Cauchy sequence converges ({\emph{completeness}}).
        \end{itemize}
      \end{alertblock}
\end{frame}


\begin{frame}{Examples of Hilbert Spaces}
  \begin{itemize}
    \item {\emph{Any finite-dimensional vector space with an inner product}} is a Hilbert space when equipped with the norm $\norm{\cdot} = \sqrt{\ip{\cdot, \cdot}}$.
    \item Let $\ell^2$ be the vector space of {\emph{complex square-summable sequences}}: %all sequences $(x_n)$ with
    % \begin{equation*}
    %   \sum_{j=1}^{\infty} \abs{x_j}^2 < \infty.
    % \end{equation*}
    \begin{equation*}
      \ell^2 = \set{(x_j) \,:\, \text{complex sequences}\,\, (x_j) \,\, \text{with}\,\, \sum_{j=1}^{\infty} \abs{x_j}^2 < \infty}.
    \end{equation*}
    The following inner product
    \begin{equation*}
      \ip{x, y} = \sum_{j = 1}^{\infty} x_j \overline{y_j}
    \end{equation*}
    makes $\ell^2$ into a Hilbert space when equipped with the norm $\norm{\cdot} = \sqrt{\ip{\cdot, \cdot}}$.
  \end{itemize}
\end{frame}

\begin{frame}{Operators}
  We'll be interested in studying a specific type of linear map on $\HS$.
  \begin{alertblock}{Bounded operator}
    An {\emph{operator}} is a linear map $T: \HS \to \HS$: for all $x, y \in \HS$ and $\mu, \lambda \in \C$,
    \begin{equation*}
      T(\mu x + \lambda y) = \mu T(x) + \lambda T(y).
    \end{equation*}

    \medskip

    We say that $T$ is {\emph{bounded}} if there exists some $M > 0$ such that for every $x \in \HS$,
    \begin{equation*}
      \norm{T(x)} \leq M \norm{x}.
    \end{equation*}

  \end{alertblock}
\end{frame}

\begin{frame}{Boundedness $\iff$ continuity}
  \begin{block}{Boundedness is equivalent to continuity}
    An operator $T: \HS \to \HS$ is bounded if and only if $T$ is continuous.
  \end{block}

  \bigskip

  \begin{block}{Finite Hilbert Spaces mean bounded operators}
    If $\HS$ is finite-dimensional then every operator $T: \HS \to \HS$ is bounded.
  \end{block}
\end{frame}

% \begin{frame}{The Operator Norm}
%   The set of bounded operators is a subspace of the set of all linear maps $L: \HS \to \HS$, and has a useful norm:
%   \begin{alertblock}{Operator norm}
%     For a bounded operator $T: \HS \to \HS$, we define the {\emph{operator norm}} to be
%     \begin{equation*}
%       \op{T} \coloneqq \text{inf}\set{M > 0 \, : \, \norm{T(x)} \leq M\norm{x} \text{ for every } x \in \HS}.
%     \end{equation*}
%   \end{alertblock}
%
%   This norm always makes the space of bounded operators complete!
% \end{frame}

\begin{frame}{Adjoints and their existence}
  \begin{alertblock}{Adjoints}
    For any operator $T: \HS \to \HS$, we say that $T^*: \HS \to \HS$ is its {\emph{adjoint}} if for all $x, y \in \HS$,
    \begin{equation*}
      \ip{x, T(y)} = \ip{T^*(x), y}.
    \end{equation*}
    $T$ is {\emph{self-adjoint}} if $T = T^*$.
  \end{alertblock}

  \begin{block}{Adjoints always exist uniquely}
    If $T: \HS \to \HS$ is an operator, then there exists a unique adjoint $T^*$.
  \end{block}
\end{frame}


\begin{frame}{Examples of an adjoint}
  \begin{itemize}
    \item The {\emph{identity map}} and the {\emph{zero map}} are self-adjoint.
    \item For any {\emph{bounded}} operator $T: \HS \to \HS$, the following are self-adjoint:\begin{itemize}
      % \item $\frac{1}{2}(T + T^*)$,
      % \item $\frac{1}{2i}(T - T^*)$,
      \item $T^*T$,
      \item $T^*T$.
    \end{itemize}
    \item The {\emph{right-shift operator}} $R: \ell^2 \to \ell^2$, \[
      R(x) = (0, x_1, x_2, \cdots)
    \] has the {\emph{left-shift operator}} as its adjoint: $L: \ell^2 \to \ell^2$, \[
      L(y) = (y_2, y_3, y_4, \cdots)
    \].
  \end{itemize}
\end{frame}

\begin{frame}{Compact Operators}
  \begin{alertblock}{Compact Operator}
    An operator $T: \HS \to \HS$ is {\emph{compact}} if \\

    \begin{center}
      For any bounded sequence $(x_j)$ in $\HS$, \\

      $(T(x_j))$ has a convergent subsequence.
    \end{center}
  \end{alertblock}

  \begin{block}{Compact operators are bounded}
    If $T: \HS \to \HS$ is compact, then it is bounded.

    The set of compact operators is a subspace of the set of bounded operators.
  \end{block}

  \begin{block}{Compactness and adjoints}
    $T: \HS \to \HS$ being compact is equivalent to $T^*$ being compact.
  \end{block}
\end{frame}

\begin{frame}{Example of a compact operator}
  % Recall that $\ell^2$ is the space
  % \begin{equation*}
  %   \ell^2 = \set{(x_j) \,:\, \text{complex sequences}\,\, (x_j) \,\, \text{with}\,\, \sum_{j=1}^{\infty} \abs{x_j}^2 < \infty}.
  % \end{equation*}
  % and is a Hilbert space with respect to the inner product
  % \begin{equation*}
  %   \ip{x, y} = \sum_{j = 1}^{\infty} x_j \overline{y_j}.
  % \end{equation*}
  % For $n \in \N$, the operator $T_n: \ell^2 \to \ell^2$,
  % \begin{equation*}
  %   T_n(x) = \left( x_1, x_2, x_3, \cdots, x_k, 0, 0, 0, \cdots \right)
  % \end{equation*}
  % is compact.
  For any Hilbert space $\HS$, then for all $x \in \HS$, there exist {\emph{orthogonal}} vectors $y, z \in \HS$ with
    \begin{equation*}
      x = y + z.
    \end{equation*}
  Let $P: \HS \to \HS$ be the {\emph{orthogonal projection}}\[P(x) = y.\] Then,
  \begin{itemize}
  \item $P$ is {\emph{self-adjoint}},
  \item {\emph{$P$ is compact if and only if $\HS$ is finite-dimensional.}}
  \end{itemize}

\end{frame}

\begin{frame}{Matrix Diagonalisation}
  From MATH220, we know that {\emph{real symmetric matrices}} ($X^T = X$) can be {\emph{diagonalised}}:

  \begin{block}{Spectral Theorem for Real Symmetric Matrices}
    Let $X \in \M_n(\R)$ be a matrix. Then:
    \begin{itemize}
      \item $X$ has an orthonormal basis $\iff$ $X$ is symmetric.
      \item If $X$ has an orthonormal basis, let $P$ be the matrix whose columns form an orthonormal basis of $X$ and $D$ be the diagonal matrix of eigenvalues in the same order as $P$. Then,
      \begin{equation*}
        X = P D P^T.
      \end{equation*}
    \end{itemize}
  \end{block}
  \vspace{-0.5cm}
  Can this be generalised to our operators?
\end{frame}

% \begin{frame}{Spectral Theory}
%     \begin{alertblock}{Spectrum of an operator}
%       The {\emph{spectrum}} of an operator $T: \HS \to \HS$ is the set
%       \begin{equation*}
%         \sigma(T) = \set{\lambda \in \C \,:\,  T - \lambda I \,\text{is not invertible.} }
%       \end{equation*}
%     \end{alertblock}
%     If $\lambda$ is an {\emph{eigenvalue}} for $T$, then
%     \begin{equation*}
%       \lambda \in \sigma(T).
%     \end{equation*}
% \end{frame}
%
% \begin{frame}{Example of a spectrum}
%   Let $T: \C^3 \to \C^3$ be the operator
%   \begin{equation*}
%     T(x) = \begin{pmatrix}
%               1 & 0 & 1 \\
%               0 & 2 & 0 \\
%               3 & 0 & 3
%             \end{pmatrix}
%             x.
%   \end{equation*}
%   Then $\sigma(T) = \set{0, 2, 4}$.
%
%   \medskip
%
%   These are the eigenvalues of $T$, which isn't a coincidence:
%   \begin{center}{\emph{In finite dimensions}}, the spectrum of an operator will always be the eigenvalues!\end{center}
% \end{frame}
%
% \begin{frame}{Properties of the spectrum}
%   \begin{block}{The spectrum is on-empty, closed, bounded}
%     For any operator $T: \HS \to \HS$, the spectrum $\sigma(T)$ is:
%     \begin{center}
%       \begin{minipage}{0.3\textwidth}
%         \begin{itemize}
%           \item non-empty,
%           \item closed,
%           \item bounded.
%         \end{itemize}
%       \end{minipage}
%     \end{center}
%   \end{block}
%
%   \begin{block}{Spectrum of the adjoint}
%     For an operator $T: \HS \to \HS$,
%     \begin{equation*}
%       \sigma(T^*) = \set{\bar{\lambda} \,:\, \lambda \in \sigma(T)}.
%     \end{equation*}
%
%     Note: If $T$ is self-adjoint, $\sigma(T) \subset \R$.
%   \end{block}
% \end{frame}


\begin{frame}{The Spectral Theorem}
  \begin{block}{The Spectral Theorem for Compact Self-Adjoint Operators}
  Suppose $T: \HS \to \HS$ is a {\emph{compact}} and {\emph{self-adjoint}} operator. Then,
  \begin{itemize}
    \item $\HS$ has an orthonormal basis $(v_n)$ of {\emph{eigenvectors}} of $T$.
    \item For $n \in \N$ or $n = \infty$, \[T(x) = \sum_{j=1}^{n}\lambda_j \ip{x, v_j}v_j,\] where $\lambda_j$ is the eigenvalue of the eigenvector $v_j$.

    \medskip

    If $n = \infty$, then $\lambda_j \to 0$ as $j \to \infty$.
  \end{itemize}
\end{block}
\end{frame}

\begin{frame}[plain]
  \begin{center}
    {\Huge Thanks for listening!}

    \vspace{1cm}

    {\huge Any questions?}
  \end{center}
\end{frame}

\begin{frame}[allowframebreaks]{References}}

  \bibliography{references}
  \bibliographystyle{abbrv}

\end{frame}
\end{document}
