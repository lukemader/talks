\usepackage[utf8]{inputenc}\usepackage[a4paper,
                                      %right = 3cm
                                      lines = 42
                                      ]{geometry}
\setlength{\textwidth}{15cm}
\usepackage[export]{adjustbox}
\usepackage{amsmath,
            amssymb,
            gensymb,
            fancyhdr,
            esint,
            tcolorbox,
            epigraph,
            mathtools,
            titlesec,
            etoolbox,
            lastpage,
            scrextend,
            framed,
            enumitem,
            lipsum}
\usepackage[colorlinks=true,
            linkcolor=black,
            anchorcolor=black,
            citecolor=black,
            filecolor=black,
            menucolor=black,
            runcolor=black,
            urlcolor=black]{hyperref}

%\usepackage[nottoc,notlot,notlof]{tocbibind} % add bib to contents page
\usepackage[backend=biber,
            style=alphabetic,
           ]{biblatex}
\addbibresource{references.bib}


\usepackage{textcomp} % i forget this package's purpose

\setlength{\parindent}{0pt}

%headers
\usepackage{fancyhdr}
\usepackage{ifthen}

\pagestyle{fancy}
\fancyhead{}
\fancyhead[L]{\ifthenelse{\isodd{\value{page}}}{\leftmark}{\rightmark}}
\fancyhead[R]{\thepage}
\renewcommand{\headrulewidth}{0pt}
\fancyfoot{}


\usepackage{amsthm}
\theoremstyle{definition}
\newtheorem{theorem}{Theorem}
\numberwithin{theorem}{section}
\numberwithin{theorem}{subsection}
\newtheorem{definition}[theorem]{Definition}
\newtheorem{lemma}[theorem]{Lemma}
\newtheorem{corollary}[theorem]{Corollary}
\newtheorem{proposition}[theorem]{Proposition}
\newtheorem{example}[theorem]{Example}
\newtheorem{remark}[theorem]{Remark}
\newtheorem{axiom}{Axiom}

\newcommand{\set}[1]{\left\{#1\right\}}
\newcommand{\N}{\mathbb{N}}
\newcommand{\R}{\mathbb{R}}
\newcommand{\Z}{\mathbb{Z}}
\newcommand{\Q}{\mathbb{Q}}
\newcommand{\C}{\mathbb{C}}
\newcommand{\M}{\mathrm{M}}
\newcommand{\F}{\mathbb{F}}
\newcommand{\HS}{\mathcal{H}}
\newcommand{\KS}{\mathcal{K}}
\newcommand{\G}{\operatorname{G}}

\newcommand{\eps}{\varepsilon}
\newcommand{\id}{\text{id}}
\newcommand{\dist}[1]{\text{dist}\left(#1\right)}
\newcommand{\ip}[1]{\left\langle #1\right\rangle}
\newcommand{\conjugate}[1]{\mkern 1.5mu\overline{\mkern-1.5mu#1\mkern-1.5mu}\mkern 1.5mu}
\newcommand{\vect}[1]{\boldsymbol{#1}}


\DeclarePairedDelimiter\abs{\lvert}{\rvert}%
\DeclarePairedDelimiter\norm{\lVert}{\rVert}%

% Swap the definition of \abs* and \norm*, so that \abs
% and \norm resizes the size of the brackets, and the
% starred version does not.
\makeatletter
\let\oldabs\abs
\def\abs{\@ifstar{\oldabs}{\oldabs*}}
%
\let\oldnorm\norm
\def\norm{\@ifstar{\oldnorm}{\oldnorm*}}
\makeatother


\newcommand{\op}[1]{\norm{#1}_\text{op}}



% bra-ket notation
% uses package mathtools
\DeclarePairedDelimiter\bra{\langle}{\rvert}
\DeclarePairedDelimiter\ket{\lvert}{\rangle}
\DeclarePairedDelimiterX\braket[2]{\langle}{\rangle}{#1 \vert #2}
\DeclarePairedDelimiterX\ketbra[2]{\delimsize\vert}{\delimsize\vert}{#1 \rangle\langle #2}

\newcommand*{\rvec}[1]{\left( #1\right)}
\newcommand*{\tr}[1]{\operatorname{tr}\left(#1\right)}
\newcommand*{\range}[1]{\operatorname{range}\left(#1\right)}
\renewcommand*{\ker}[1]{\operatorname{ker}\left(#1\right)}
\newcommand{\dom}[1]{\operatorname{Dom}\left(#1\right)}
