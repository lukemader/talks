In this last section, we consider the quantum harmonic oscillator. This is a model of a particle moving in $\R$ of utmost importance to physicists, as it simplifies a lot of complicated systems. Our aim is to give a full description of the eigenvectors and eigenvalues of the Hamiltonian for this particle, which we take from {\cite[Chapter 11]{Hall2013}}. The reasoning for finding the eigenvalues is explained in {\cite[p.14]{Hall2013}}; the eigenvalues here represent the possible energy levels it can have. Our investigation is split into two parts; the first part involves considering the commutation properties of our operators, and the second involves considering a differential equation.

\subsubsection{The algebraic method: considering commutation properties}

In this section, our aim is to find out as much as we can about the eigenvalues of the Hamiltonian operator for the quantum harmonic oscillator using only the commutatation relation $[X, P] = i \hbar I$. Our Hilbert space for this quantum system is $L^2 (\R)$, and we start by stating the Hamiltonian of our system.


%
% and we will start by showing that the domain for the Hamiltonian for our system is dense.
%
% \begin{proposition}
%   The subspace \(D_n = \text{span} \set{x^{\alpha}e^{-x^2/2} \, : \, \alpha \in \N_{0}^{n}} \subset L^2(\R^n)\) is dense.
% \end{proposition}
% \begin{proof}%[{\cite[Lemma 8.3]{teschl}}]
%   Teschl Lemma 8.3.
% \end{proof}


\begin{definition}[{\cite[Section 11.2]{Hall2013}}]
  The Hamiltonian for the quantum harmonic oscillator is the operator $H: \mathcal{S}(\R) \to L^2(\R)$ given by
    \begin{equation*}
      H = \frac{1}{2m} \left( P^2 + (m \omega X)^2 \right),
    \end{equation*}
    where $\omega > 0$ is some positive constant called the {\emph{angular momentum}} and $X$ and $P$ are the one-dimensional position and momentum operators.
\end{definition}

 In order to perform our analysis, it is useful to introduce two operators called the {\emph{lowering}} and {\emph{raising}} operators.

 \begin{definition}[{\cite[Section 11.2]{Hall2013}}]
   We define the {\emph{lowering operator}}, $a: \mathcal{S}(\R) \to L^2(\R)$, and the {\emph{raising operator}}, $a^*: \mathcal{S}(\R) \to L^2(\R)$, to be the operators
   \begin{equation*}
     a = \frac{1}{\sqrt{2\hbar m \omega}} \left( m \omega X + iP \right)
     \quad \text{and} \quad
     a^* = \frac{1}{\sqrt{2\hbar m \omega}} \left( m \omega X - iP \right).
   \end{equation*}
\end{definition}
\begin{remark}\label{lbl_remark_lowering_raising_maps_schwartz_space_to_itself}
  As we noticed in Remark \eqref{lbl_position_momentum_map_SR_to_SR}, the position and momentum operators map $\mathcal{S}(\R)$ into itself. As $\mathcal{S}(\R)$ is a vector space, it therefore follows that $a$, $a^*$, $a^*a$, and $aa^*$ map $\mathcal{S}(\R)$ into itself.
\end{remark}

\begin{proposition}[{\cite[Proof of Theorem 11.4]{Hall2013}}]\label{lbl_prop_lowering_raising_symmetric}
  Let $\phi$ and $\psi$ be vectors in $\mathcal{S}(\R)$. Then we have that
  \begin{enumerate}[label=(\alph*)]
    \item $\ip{\phi, a\psi} = \ip{a^* \phi, \psi}$.
    \item The operator $a^*a$ is symmetric and positive on $\mathcal{S}(\R)$.
  \end{enumerate}
\end{proposition}
\begin{proof}
  The proof of this result follows very quickly from the properties of the inner product. For part (a), all that is needed is to start with one-side and use the linearity and conjugate-linearity of the inner product. For part (b), symmetry follows by using part (a) and the conjugate symmetry of the inner product, and positivity follows the calculation of symmetry and the positive semi-definiteness of the norm.  We therefore omit the verification of this result here; the interested reader can find the full details in Section \eqref{proof_lbl_prop_lowering_raising_symmetric}. We note that our method is slightly different to the argument made in the proof of {\cite[Theorem 11.4]{Hall2013}}.
\end{proof}


\begin{proposition}[{\cite[p.229]{Hall2013}}]\label{lbl_prop_quantum_harmonic_oscillator_ito_ladders}
  The Hamiltonian $H \colon \mathcal{S}(\R) \to L^2(\R)$ for the quantum harmonic oscillator is given by
  \begin{equation*}
    H = \hbar \omega \left(a^* a + \frac{1}{2}I \right),
  \end{equation*}
  where $a$ and $a^*$ are also defined on $\mathcal{S}(\R)$.
\end{proposition}
\begin{proof}
  The verification of this result is a simple calculation with use of the canonical commutation relationship for $X$ and $P$, provided in Proposition \eqref{lbl_cannonical_commutation_relations}. This is the method shown in {\cite[p.229]{Hall2013}}. We therefore move the full details of the calculation to Section \eqref{proof_lbl_prop_quantum_harmonic_oscillator_ito_ladders} for the interested reader.
\end{proof}

What this result means is that to work out the eigenvalues and eigenvectors of $H$, we can just look at the eigenvectors and eigenvalues of $a^*a$. $H$ will then have the same eigenvectors, and if $\lambda$ is an eigenvalue for $a^* a$, then we have that $\hbar \omega \left(\lambda + \frac{1}{2}\right)$ is an eigenvalue for $H$. This means that all of our eigenvalue and eigenvector results about $a^*a$ can therefore be easily translated into all of the eigenvalue and eigenvector results for $H$.

\medskip

Before we can find the eigenvalues and eigenvectors of $a^*a$, the final piece of information we need to know is how $a$ and $a^*$ commute with $a^*a$.

\begin{proposition}[{\cite[p.229]{Hall2013}}] \label{lbl_raising_commutation_relationships}
  On $\mathcal{S}(\R)$, we have the commutator identities
  \begin{align*}
    [a, a^* a] = a,
    \quad \text{and} \quad
    [a^*, a^* a] = -a^*.
  \end{align*}
\end{proposition}
\begin{proof}
  The verification of this result is a simple but slightly messy calculation with use of the standard results of the commutator given in Proposition \eqref{lbl_prop_commutator_properties}. Due to our time constraints, we therefore move this proof to Section \eqref{proof_lbl_raising_commutation_relationships}, which we do in full detail. We note that this result is stated on {{\cite[p.229]{Hall2013}}}, with calculation only for $[a,a^*]$.
\end{proof}

We now are able to describe the eigenvalues for $a^*a$. Recall that we pointed out in Remark \eqref{lbl_remark_lowering_raising_maps_schwartz_space_to_itself} that $a$, $a^*$, and $a^*a$ all map $\mathcal{S}(\R)$ into itself.
\begin{proposition}[{\cite[Proposition 11.1]{Hall2013}}]\label{lbl_aastar_lowers_raises}
  Suppose that $a^*a$ is defined on $\mathcal{S}(\R)$ and has an eigenvector $\psi$, and let $\lambda$ be the corresponding eigenvalue to $\psi$. Then,
  \begin{enumerate}[label=(\alph*)]
    \item $a^*a (a\psi) = (\lambda - 1)a\psi$.
    \item $a^*a (a^* \psi) = (\lambda + 1)a^*\psi$.
  \end{enumerate}
\end{proposition}
\begin{proof}
    The proof of this result is a very simple calculation by looking at the left-hand sides of part (a) and (b) and then applying the commutation results we saw in Proposition \eqref{lbl_raising_commutation_relationships}. Due to our time constraints, we therefore move this verification to Section \eqref{proof_lbl_aastar_lowers_raises}.
\end{proof}
\begin{remark}[{\cite[Proposition 11.1]{Hall2013}}]
  What this proposition tells us is that either $a \psi = 0$ or $a \psi$ is an eigenvector for $a^*a$ with eigenvalue $\lambda - 1$. This is why $a$ is called the lowering operator; it lowers the corresponding eigenvalue of $a^*a$ by 1. Similarly, if $a^*\psi$ is non-zero, it is an eigenvector for $a^*a$ which raises the corresponding eigenvalue of $a^*a$ by 1. An important consequence of this is that we can repeatedly apply $a$ to $\psi$ to get 0. This is very illustrative of the power of the lowering and raising operators.
\end{remark}
\begin{lemma}[{\cite[p.230]{Hall2013}}]\label{lbl_eigenvalues_aastar}
  Every eigenvalue of $a^*a$ with the domain $\mathcal{S}(\R)$ is non-negative.
\end{lemma}
\begin{proof}
  The verification of this result boils down to showing that $\lambda\ip{\psi, \psi}$ is non-negative for any eigenvector $\psi$ with corresponding eigenvalue $\lambda$.  Due to our time constraints, we therefore move this proof to Section \eqref{proof_lbl_eigenvalues_aastar} for the interested reader.
\end{proof}
\begin{proposition}[{\cite[p.230]{Hall2013}}] \label{lbl_aastar_ground_state}
  Suppose that $\psi$ is an eigenvalue for $a^*a$ with the domain $\mathcal{S}(\R)$. There then exists a non-zero eigenvector $\psi_0 \in \mathcal{S}(\R)$ such that
  \begin{align*}
    a^*a\psi_0 = 0 \quad \text{and} \quad
    a \psi_0 = 0.
  \end{align*}
  This eigenvector is an eigenvector of $a^*a$ with the lowest possible eigenvalue, and we therefore call it a {\emph{ground state}} of $a^*a$.
\end{proposition}
\begin{proof}
  This proof comes from the comments on {\cite[p.230]{Hall2013}}, which we repeat here. Let $\psi$ be an eigenvector of $a^*a$ with corresponding eigenvalue $\lambda$.

  \medskip

  Now, let $n$ be some natural number. By Proposition \eqref{lbl_aastar_lowers_raises}(a), we have that $a\psi$ is an eigenvector of $a^*a$ with eigenvalue $\lambda - 1$. By applying Proposition \eqref{lbl_aastar_lowers_raises}(a) again, this means that $a^2 \psi$ is an eigenvector of $a^*a$ with eigenvalue $\lambda - 2$. Repeating this process $n$ times shows us that $a^n \psi$ is an eigenvector of $a^*a$ with eigevalue $\lambda - n$. Now, if $a^n \psi$ for all $n \in \N$ is non-zero, then at some point we would have to get a negative eigenvalue, as each application of $a$ gives an eigenvalue whose value decreases by 1 each time. However, Lemma \eqref{lbl_eigenvalues_aastar} tells us that the eigenvalues of $aa^*$ are always non-negative, so at some point we have to get an eigenvector with 0 as its eigenvalue. This means that there has to be a $N \in \N$ such that $a^N \psi \neq 0$ but $a^{N+1} \psi = 0$. We therefore see that $\psi_0 = a^{N}\psi$ gives us our result.
\end{proof}
Ground states are useful as they allow us to find an orthogonal chain of eigenvectors for $a^*a$, for which we can describe how $a$ and $a^*$ act on.

\begin{theorem}[{\cite[Theorem 11.2]{Hall2013}}]\label{lbl_oscillator_chains}
  Let $a^*$ and $a$ be defined on $\mathcal{S}(\R)$, and suppose that $\psi_0$ is a unit vector such that $a\psi_0 = 0$. For all $n \in \N_0$, let $\psi_n = (a^*)^n \psi_0$. For all $n,m \in \N_0$, we then have that:
  \begin{enumerate}[label=(\alph*)]
    \item $a^*\psi_n = \psi_{n+1}$.
    \item $a^* a \psi_n = n \psi_n$.
    \item $\ip{\psi_n, \psi_m} = n!\delta_{n,m}$.
    \item $a\psi_{n + 1} = (n+1)\psi_n$.
  \end{enumerate}
\end{theorem}
\begin{proof}
  We take this proof from the proof of {\cite[Theorem 11.2]{Hall2013}}. The first statement follows from the definition of $\psi_{n+1}$, as \[a^*\psi_n = a^* (a^*)^n \psi = (a^*)^{n+1} = \psi_{n+1}.\]

  The second statement is easily seen by induction. Recall that $\psi_0$ is the vector that satisfies the equation $a^* a \psi_0 = 0 = 0 \psi_0$. Now, assume that there is some $k \in \N_0$ such that $a^* a \psi_k = k \psi_k$, and note that this specifically means that $\psi_k$ is an eigenvector with eigenvalue $k$. With use of the first statement and the result in Proposition \eqref{lbl_aastar_lowers_raises} that $a^* a (a^* \psi) = (\lambda + 1) a^* \psi$ for any eigenvector $\psi$ of $a^*a$ with eigenvalue $\lambda$, we then see that
  \begin{align*}
      a^*a \psi_{k+1}
      =
      a^* a (a^* \psi_{k})
      =
      (k+1)a^* \psi_k
      =
      (k+1)\psi_{k+1},
  \end{align*}
  so by induction we see that the second statement is true for all $n \in \N$. Note that this means that the point spectrum $\sigma_p(a^*a)$ satisfies $\sigma_p(a^*a) \subset \N_0$ as $\psi_0$ is the eigenvector with the lowest eigenvalue (which is 0), and every other eigenvector's eigenvalue is a natural number.

  \medskip

  For the third statement, recall from Propoposition \eqref{lbl_prop_sym_op_eigenvalues_orthogonal} that the eigenvectors of a symmetric operator are mutually orthogonal. As we have already shown that $a^*a$ is symmetric in Proposition \eqref{lbl_prop_lowering_raising_symmetric}, this therefore means that if $n \neq m$, we have that $\ip{\psi_n, \psi_m} = 0 = n!\delta_{n,m}$ as $\delta_{n,m} = 0$. If $n = m$, we show this by induction. For $n = 0$, $\ip{\psi_0, \psi_0} = \norm{\psi}^2 = 1$ as we suppose that $\psi_0$ is a unit vector. Now, suppose that $\ip{\psi_n, \psi_n} = n!$. The proof of Proposition \eqref{lbl_raising_commutation_relationships} shows us that $[a, a^*] = I$, meaning that $aa^* = a^*a + I$, we see that
  \begin{align*}
    \ip{\psi_{n+1,} \psi_{n+1}}
    &=
    \ip{a^* \psi_n, a^* \psi_n} \\
    &=
    \ip{a a^* \psi_n, \psi_n} \\
    &=
    \ip{(a^*a + I) \psi_n, \psi_n} \\
    &=
    \ip{n\psi_n + \psi_n, \psi_n} \\
    &=
    (n+1)\ip{\psi_n, \psi_n} \\
    &=
    (n+1)!,
  \end{align*}
  so by induction we get our result. For the fourth and final statement, part (b) tells us that
  \begin{align*}
    a \psi_{n+1}
    = aa^*\psi_{n}
    = (a^*a + I) \psi_n
    = (n+1) \psi_n.
  \end{align*}
\end{proof}
The vectors $\psi_n$ in the above result are called the {\emph{excited states}} of the quantum harmonic oscillator. Part (b) tells us that these are the eigenvectors for $a^*a$.

\subsubsection{The analytical methods: considering differential equations}

In the last section, we found out a remarkable amount about the eigenvalues of $a^*a$, and therefore of our Hamiltonian, using just the commutation relation $[X, P] = i \hbar I$. Our analysis is, however, incomplete, as we would like an explicit formula for $\psi_n$ and to know if they are dense in $L^2(\R)$. In order to make our calculations simplier, we will use the following scale of distance, $D$, which allows us to normalise our position variable, which we denote by $\tilde{x}$:

\begin{equation*}
  D = \sqrt{\frac{\hbar}{m\omega}} \qquad \text{and} \qquad \tilde{x} = \frac{x}{D}.
\end{equation*}

This normalised position variable makes our life easier as we can express our raising and lowering operators, $a^*$ and $a$, without the constants $m$, $\omega$, and $\hbar$.

\begin{proposition}[{\cite[p.232]{Hall2013}}]\label{lbl_prop_2_3_12}
  The operators $a$ and $a^*$ when defined on $\mathcal{S}(\R)$ can be expressed as
  \begin{align*}
    a   &= \frac{1}{\sqrt{2}}\left( \tilde{x} + \frac{d}{d\tilde{x}} \right) \quad \text{and} \quad
    a^* = \frac{1}{\sqrt{2}}\left( \tilde{x} - \frac{d}{d\tilde{x}} \right).
  \end{align*}
\end{proposition}
\begin{proof}
  The proof for this is a simple calculation involving a change of variables. We therefore move the calculation, which we perform in full detail, to Section \eqref{proof_lbl_prop_2_3_12} for the interested reader.
\end{proof}

With this simplification of our lowering and raising operators, we can easily figure out an equation for our ground states $\psi_0$.

\begin{lemma}[{\cite[Proposition A.22]{Hall2013}}]\label{lbl_oscillator_hard_integral}
  For any non-negative $\alpha \in \R$ and for all $\beta \in \C$, we have that
  \begin{equation*}
    \int_{-\infty}^{\infty} \exp\left(-\frac{x^2}{a}\right) \exp\left(bx\right) \,\mathrm{d}x
    =
    \sqrt{2\pi a}\exp\left(ab^2\right).
  \end{equation*}
\end{lemma}
\begin{proof}
  In order to stay focused, we omit the proof and take this for granted.
\end{proof}

\begin{theorem}[{\cite[p.232]{Hall2013}}]\label{lbl_oscillator_general_formula_ground_state}
  On the domain $\mathcal{S}(\R)$, the equation $a \psi_0 = 0$ has solution
  \begin{equation*}
    \psi_0 (x) = \left(\frac{m \omega}{\pi \hbar}\right)^{\frac{1}{4}} \exp\left( -\frac{m \omega}{2 \hbar}x^2 \right).
  \end{equation*}
\end{theorem}
\begin{proof}
  This result comes from the comments on {\cite[p.232]{Hall2013}}, which has a miscalculated constant fixed in the corrections of the book. The proof is a simply solving a first order differential equation, so we move the calculations to Section \eqref{proof_lbl_oscillator_general_formula_ground_state} for the interested reader. The method is as follows: as $a \psi_0 = 0$, we get a first order differential equation $\psi^\prime_0 (\tilde{x}) + \tilde{x}\psi_0(\tilde{x}) = 0$. By using the method of integrating factors and then substituting in the definition of $\tilde{x}$ to get it in terms of $x$, we get our required result. The interested reader is encouraged to see the full details in Section \eqref{proof_lbl_oscillator_general_formula_ground_state}.
\end{proof}


Now that we have an explicit equation for our ground states $\psi_0$, we can find the excited states $\psi_n$ of our system. We first introduce a family of polynomials, which are more simple form of the {\emph{Hermite polynomials}}.

\begin{definition}[{\cite[Theorem 11.3]{Hall2013}}]
  The family of polynomials $H_n$, where $n \in \N_0$ is the degree of the polynomial, is defined inductively by
  \begin{align*}
    H_0 (\tilde{x}) = 1, \qquad
    H_{n + 1} (\tilde{x}) = \frac{1}{\sqrt{2}} \left( 2\tilde{x}H_n(\tilde{x}) - \frac{d}{d\tilde{x}} H_n{\tilde{x}} \right).
  \end{align*}
\end{definition}

\begin{theorem}[{\cite[Theorem 11.3]{Hall2013}}]\label{lbl_thrm_all_excited_states_of_qho}
  The excited states of the quantum harmonic oscillator with the domain $\mathcal{S}(\R)$, $\psi_n$, are given by
  \begin{equation*}
    \psi_n = H_n \psi_0.
  \end{equation*}
\end{theorem}
\begin{proof}
  This proof is a very simple, so we omit it here due to our time considerations. The method is to prove it by induction. The $n = 0$ case holds trivially as $H_1(\tilde{x}) = 1$. To verify that it holds for $n = k + 1$ if it holds for $n - 1 = k$, one needs to use the relation $\psi_{k+1} = a^* \psi_k$ that we established in Theorem \eqref{lbl_oscillator_chains}, and the general solution for $\psi_0(\tilde{x})$ which we established in Theorem \eqref{lbl_oscillator_general_formula_ground_state}. Due to the simplicity and our time constraints, we omit this proof; the reader can find the full details in Section \eqref{proof_lbl_thrm_all_excited_states_of_qho}.
\end{proof}

We now finally give the result that the excited states $\psi_n$ provide an orthonormal basis for $L^2(\R)$, which relies on the following lemma.

\begin{lemma}[{\cite[Lemma 11.5]{Hall2013}}]\label{lbl_lemma_excited_states_ONB}
  Let $\alpha \in \C$ be any scalar. Then, the partial sums of the series
  \begin{equation*}
    \sum_{n=0}^{\infty} \frac{\alpha^n \tilde{x}^n}{n!}\exp\left(-\frac{\tilde{x}^2}{2}\right)
  \end{equation*}
  converge in $L^2(\R)$ to $\exp\left(\alpha \tilde{x}\right)\exp\left(-\frac{\tilde{x}^2}{2}\right)$.
\end{lemma}
\begin{proof}
  The proof of this result relies on a simple method but with some messy calculations. We therefore move this proof to Section \eqref{proof_lbl_lemma_excited_states_ONB}, which is a showcase of the proof from {\cite[Lemma 11.5]{Hall2013}}. The method involves noting that we have the pointwise convergence
  \begin{equation*}
    \sum_{n=0}^{\infty} \left(\frac{(\alpha \tilde{x})^n}{n!} \exp\left( -\frac{\tilde{x}^2}{2} \right)\right)
    =
    \exp\left(\alpha \tilde{x}\right) \exp\left( -\frac{\tilde{x}^2}{2} \right).
  \end{equation*}
  This means that the tail of the sum on the left-hand side must pointwise converge to 0 as we take a limit. By finding an appropriate bound on the tail of the sum, we see that it is square-integrable. After some fiddling around with the expression of our partial sums and their potential limit point, our result then falls through by the dominated convergence theorem, Theorem \eqref{lbl_thrm_dominated_convergence}. The interested reader is encouraged to look at the full details in Section \eqref{proof_lbl_lemma_excited_states_ONB}.
\end{proof}

\begin{theorem}[{\cite[Theorem 11.4]{Hall2013}}]\label{lbl_thrm_excited_states_onb}
  The excited states of the quantum harmonic oscillator with the domain $\mathcal{S}(\R)$, $\psi_n(x)$, make up an orthonormal basis for $L^2(\R)$.
\end{theorem}
\begin{proof}
  We showcase this proof from the proof of {\cite[Theorem 11.4]{Hall2013}}. We start by recalling that $\psi_n$ are mutually orthogonal by Theorem \eqref{lbl_oscillator_chains}. Now, let $V = \text{span}\set{\psi_n}_{n \in \N}$. As $\psi_n = H_n \psi_0$ and $H_n$ is just a polynomial, we have that, for every $\psi \in V$, there is some polynomial $p$ of degree n such that
  \begin{equation*}
    \psi(\tilde{x})
    =
    p(\tilde{x})\exp \left( -\frac{\tilde{x}^2}{2} \right).
  \end{equation*}
  Now, for any value of our constants $\omega$ and $m$, by Lemma \eqref{lbl_lemma_excited_states_ONB} we have that $\exp\left(i m\omega^2 \tilde{x}\right) \exp\left(-\frac{\tilde{x}^2}{2}\right)$ belongs to the closure of $V$ in $L^2(\R)$. Therefore, we have that for every  $\psi \in V^\perp$,
  \begin{equation*}
    \int_{-\infty}^{\infty}
    \psi(\tilde{x}) \exp\left(i m\omega^2 \tilde{x}\right) \exp\left(-\frac{\tilde{x}^2}{2}\right)
    \,\mathrm{d}\tilde{x}
    =
    0.
  \end{equation*}
  This next step relies on the Fourier transform, which so far we have ommitted. We repeat this step directly from the proof of {\cite[Theorem 11.4]{Hall2013}} for the purpose of completeness, and take for granted that the consequences are true. Now, as both $\exp\left( -\frac{\tilde{x}^2}{2} \right)$ and $\psi(\tilde{x})$ are elements of $L^2(\R)$, the Cauchy-Schwarz inequality tells us that $\exp\left( -\frac{\tilde{x}^2}{2} \right) \psi(\tilde{x})$ is an element of $L^1(\R)$. it is also not hard to notice that $\exp\left( -\frac{\tilde{x}^2}{2} \right)$ is an element of $L^\infty (\R)$. This is the space of bounded functions $f \colon \R \to \C$ with the supremum norm, defined in Remark \eqref{lbl_remark_general_L_p_spaces}. We now notice that
  \begin{align*}
    \norm{\exp\left( -\frac{\tilde{x}^2}{2} \right) \psi(\tilde{x})}_{L^2(\R)}^2
    &=
    \int_{\R} \abs{\exp\left( -\frac{\tilde{x}^2}{2} \right) \psi(\tilde{x})}^2 \,\mathrm{d}\tilde{x} \\
    &\leq
    \int_{\R} \norm{\exp\left( -\frac{\tilde{x}^2}{2} \right)}_{\infty}^{2} \abs{\psi(\tilde{x})} \,\mathrm{d}\tilde{x} \\
    &=
    \norm{\exp\left( -\frac{\tilde{x}^2}{2} \right)}_{\infty}^2 \norm{\psi(\tilde{x})}_{L^2(\R)}^2,
  \end{align*}
  meaning that $\exp\left( -\frac{\tilde{x}^2}{2} \right) \psi(\tilde{x})$ is also an element of $L^2(\R)$. Our above integral of
  \begin{equation*}
    \int_{-\infty}^{\infty}
    \psi(\tilde{x}) \exp\left(i m\omega^2 \tilde{x}\right) \exp\left(-\frac{\tilde{x}^2}{2}\right)
    \,\mathrm{d}\tilde{x}
    =
    0
  \end{equation*}
   therefore means that the $L^2(\R)$ Fourier transform of $\exp\left( -\frac{\tilde{x}^2}{2} \right) \psi(\tilde{x})$ is identically zero, which by the Plancherel theorem (which can be found in {\cite[A.19]{Hall2013}} or with proof in {\cite[Theorem 7.5]{teschl}}) means it must be the zero element of $L^2(\R)$. Therefore, $\psi(\tilde{x}) = 0$ almost everywhere, which gives us that $V$ is dense in $L^2(\R)$ as $V^\perp = \set{0}$. Definition \eqref{lbl_def_onb} tells us that $V$ being dense is equivalent to $V$ being an orthonormal basis for $L^2(\R)$.
\end{proof}

Our final result tells us that the Hamiltonian for the quantum harmonic oscillator is essentially self-adjoint. By Proposition \eqref{lbl_prop_esa_has_closure_unique_sa_extension}, this means that our Hamiltonian has an unique self-adjoint extension, and our work here is done.

\begin{corollary}[{\cite[p.505]{moretti}}]
  The Hamiltonian $H$ and the operator $a^*a$ for the quantum harmonic oscillator are essentially self-adjoint on $\mathcal{S}(\R)$.
\end{corollary}
\begin{proof}
  As {\cite[p.505]{moretti}} also observes, we have already done all the work needed for this result. Theorem \eqref{lbl_thrm_excited_states_onb}, which tells us that the excited states of the quantum harmonic oscillator, $\psi_n(x)$, form an orthonormal basis for $L^2 (\R)$. Proposition \eqref{lbl_aastar_lowers_raises}, Proposition \eqref{lbl_aastar_ground_state}, and Theorem \eqref{lbl_oscillator_chains} gave us a full description of the excited states and that they are the eigenvectors of $a^*a$. As they form an orthonormal basis for $L^2(\R)$, by Corollary \eqref{lbl_corollary_sym_with_eigenvectors_onb_is_esa}, $a^*a$ is essentially self-adjoint on $\mathcal{S}(\R)$. By our comments after Proposition \eqref{lbl_prop_quantum_harmonic_oscillator_ito_ladders}, the eigenvectors of $H$ are the same as the eigenvectors for $a^*a$, and if $\lambda$ is an eigenvalue of $a^*a$ then $\hbar \omega \left(\lambda + \frac{1}{2} \right)$ is an eigenvalue for $H$. This implies that $H$ is essentially self-adjoint on $\mathcal{S}(\R)$ too.
\end{proof}
