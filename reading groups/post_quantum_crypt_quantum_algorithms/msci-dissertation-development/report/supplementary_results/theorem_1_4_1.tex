\subsection{Proof of Theorem \eqref{lbl_thrm_1_4_1}}\label{proof_lbl_thrm_1_4_1}

\begin{theorem}[{\cite[Theorem 13.2]{rudin}}]
  Suppose that $T, S$, and $ST$ are operators in $\HS$. Then,
  \begin{enumerate}[label = (\alph*)]
    \item $(ST)^*$ is an extension of $T^* S^*$.
    \item If $S$ is bounded and $\dom{S} = \HS$, then $T^*S^* = (ST)^*$.
  \end{enumerate}
\end{theorem}
\begin{proof}
  We recreate this proof from {\cite[Theorem 13.2]{rudin}}. For the proof of (a), suppose that $x$ is in $\dom{ST}$ and that $y$ is in $\dom{T^*S^*}$. We must then have that $x \in \dom{T}$ and that $S^*y \in \dom{T^*}$, giving us that
  \begin{equation*}
    \ip{Tx, S^*y}
    =
    \ip{x, T^*S^*y}.
  \end{equation*}
  Also, we must have that $Tx \in \dom{S}$ and that $y \in \dom{S^*}$, so we also have that
  \begin{equation*}
    \ip{STx, y}
    =
    \ip{Tx, S^*y}.
  \end{equation*}
  Together, we get that
  \begin{equation*}
    \ip{STx, y}
    =
    \ip{x, T^*S^*y}.
  \end{equation*}
  Now, by the Cauchy-Schwarz inequality, we have that
  \begin{equation*}
    \ip{x, T^*S^*y} \leq \norm{x} \norm{T^*S^*y},
  \end{equation*}
  meaning that $x \mapsto \ip{STx, y} = \ip{x, T^*S^*y}$ is bounded, which is equivalent to it being continuous. Therefore, $y \in \dom{(ST)^*}$ as well as being in $\dom{T^*S^*}$. As the value of $(ST)^*y$ is unique, we must have that $(ST)^*y = (T^*S^*)y$ for all $y \in \dom{T^*S^*}y$. Therefore, $(ST)^*$ is an extension of $T^*S^*$.

  \medskip

  Now, for the proof of (b), suppose that $S: \HS \to \HS$ is a bounded operator and let $y \in \dom{(ST)^*}$ and let $x \in \dom{ST}$. As $S:\HS \to \HS$ is bounded, we have that $S^*: \HS \to \HS$ is bounded. In particular, it is defined on the entirety of $\HS$. As $x$ must be in $\dom{T}$ if it is in $\dom{ST}$, we have that
  \begin{equation*}
    \ip{Tx, S^*y} = \ip{STx, y} = \ip{x, (ST)^*y}.
  \end{equation*}
  Now, by the Cauchy-Schwarz inequality, we have that
  \begin{equation*}
    \ip{x, (ST)^*y} \leq \norm{x} \norm{ST^*y}.
  \end{equation*}
  This means that the linear functional $x \mapsto \ip{Tx, S^*y} = \ip{x, (ST)^*y}$ is bounded, so it is also continuous. Therefore, $S^*y \in \dom{T^*}$. This means that $y \in \dom{T^*S^*}$. Therefore, we have that $\dom{(ST)^*} \subset \dom{T^*S^*}$. By part (a), we have that $\dom{T^*S^*} \subset \dom{(ST)^*}$, giving us that $T^*S^*$ and $(ST)^*$ have the same domain and are equal on this domain.
\end{proof}
