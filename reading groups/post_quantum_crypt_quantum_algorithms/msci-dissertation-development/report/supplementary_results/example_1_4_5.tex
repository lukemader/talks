\subsection{Proof of Example \eqref{lbl_example_kreyzsig_closed_operator}}\label{proof_lbl_example_kreyzsig_closed_operator}

\begin{example}[{\cite[Problem 5, Chapter 10.3]{kreyszig}}]
  Let $T$ be the operator in $\ell^2$ introduced in Example \eqref{lbl_example_kreyzsig_not_closed_operator}, which was defined by
  \begin{align*}
    \dom{T} &= \set{x \in \ell^2 \colon \text{$x$ has finitely many non-zero elements}}, \\
    Tx &= (jx_j)_{j \in \N}.
  \end{align*}
  In Example \eqref{lbl_example_kreyzsig_not_closed_operator}, we showed that $T$ is not a closed operator. However, the operator $S$ in $\ell^2$ given by
  \begin{align*}
    \dom{S} &= \set{x \in \ell^2 \colon (jx_j)_{j \in \N} \in \ell^2}, \\
    Sx &= (jx_j)_{j \in \N}
  \end{align*}
  is a closed extension of $T$, meaning that $T$ is closable.
\end{example}
\begin{proof}
  We start by verifying that $\dom{S}$ is indeed a subspace of $\ell^2$, which, like in Example \eqref{lbl_example_kreyzsig_not_closed_operator}, is just a quick application of the subspace test. We obviously have that the zero sequence is in $\dom{S}$, as it is square-summable and $(j \times 0)_{j \in \N} = 0$ is also square-summable. Now, let $x$ and $y$ be in $\dom{S}$. We then have that
  \begin{align*}
    \sum_{j \in \N} \abs{j(x_j + y_j)}^2
    &=
    \sum_{j \in \N} j^2 \abs{x_j + y_j}^2 \\
    &\leq
    \sum_{j \in \N} j^2 \left( \abs{x_j} + \abs{y_j} \right)^2 \qquad\,\,\text{by the triangle inequality} \\
    &=
    \sum_{j \in \N} \left( j^2 \abs{x_j}^2 \right)
      +
      \sum_{j \in \N} \left( j^2 \abs{y_j}^2 \right)
      +
      2\sum_{j \in \N} \left( j^2 \abs{x_j}\abs{y_j}\right) \\
    &\leq
    \sum_{j \in \N} \left( j^2 \abs{x_j}^2 \right)
      +
      \sum_{j \in \N} \left( j^2 \abs{y_j}^2 \right)
      + 2\left(
        \sum_{j \in \N} \left( j^2 \abs{x_j}^2 \right) \sum_{j \in \N} \left( j^2 \abs{y_j}^2 \right)
      \right)^{\frac{1}{2}}
    \\&\qquad\qquad\qquad\qquad\qquad\qquad\text{by the Cauchy-Schwarz inequality} \\
    &<
    \infty,
  \end{align*}
  where the last line comes from $x$ and $y$ being in the domain of $S$. Finally, we have that for any complex number $\alpha$,
  \begin{equation*}
    \sum_{j \in \N} \abs{\alpha j x_j}^2
    =
    \alpha^2 \sum_{j \in \N} j^2 \abs{x_j}
    <
    \infty.
  \end{equation*}
  Therefore, $\dom{S}$ is a subspace of $\ell^2$ by the subspace test. Now, we clearly have that $\dom{T}$ is a subset of $\dom{S}$, as $\dom{T}$ is just the elements of $\dom{S}$ with a finite number of non-zero terms. In Example \eqref{lbl_example_kreyzsig_not_closed_operator}, we showed that $\dom{T}$ is dense; therefore, it follows that $\dom{S}$ is also dense, as it contains the entirety of $\dom{T}$. This also means that $Sx = Tx$ for all $x$ in the domain of $T$, as $S$ and $T$ have the same defining equation. Together, this means both that $S$ is an extension of $T$ and that it is densely defined.

  \medskip

  The last step we need to verify is that $S$ is closed. The solution to {\cite[Problem 5, Chapter 10.3]{kreyszig}}, which can be found in {\cite[p.667]{kreyszig}}, shows us how to do this. We introduce the operator $R \colon \ell^2 \to \ell^2$, defined by
  \begin{equation*}
    Rx = \left( \frac{1}{j}x_j \right)_{j \in \N}.
  \end{equation*}
  This is cleary defined on the entirety of $\ell^2$, as for any square-summable sequence $x$, we have that
  \begin{equation*}
    \sum_{j \in \N} \abs{\frac{1}{j} x_j}^2
    \leq
    \sum_{j \in \N} \abs{x_j}^2
    <
    \infty.
  \end{equation*}
  This inequality also shows that $R$ is bounded, as
  \begin{equation*}
    \norm{R x}^2
    =
    \sum_{j \in \N} \abs{\frac{1}{j} x_j}^2
    \leq
    \sum_{j \in \N} \abs{x_j}^2
    =
    \norm{x}^2.
  \end{equation*}
  As $R$ is a bounded operator defined on the entirety of $\ell^2$, by the closed graph theorem, Theorem \eqref{lbl_thrm_closed_graph}, it must be a closed operator. Now, $R$ is also injective, as for any $x$ and $y$ in $\ell^2$, we clearly have that
  \begin{equation*}
    \left( \frac{1}{j} x_j \right)
    =
    \left( \frac{1}{j} y_j \right)
    \qquad\iff\qquad
    \text{$x_j = y_j$ for all $j \in \N$}.
  \end{equation*}
  This means that $R$ has a defined inverse operator $R^{-1}$. Even further, as $R$ is closed, by Theorem \eqref{lbl_thrm_densely_defined_and_closability}(e), we must have that $R^{-1}$ is closed. Now, it's easy to see that for $R^{-1}$ to be the inverse of $R$, it must have the defining equation
  \begin{equation*}
    R^{-1}x = (j x_j)_{j \in \N}
  \end{equation*}
  for all $x$ in $\ell^2$. It's domain must also be the range of $R$, which we easily see is
  \begin{align*}
    \dom{R^{-1}}
    &=
    \range{R} \\
    &=
    \set{\left(\frac{1}{j}x_j \right)_{j \in \N} \colon x \in \ell^2} \\
    &=
    \set{ x \in \ell^2 \colon (j x_j)_{j \in \N} \in \ell^2}.
  \end{align*}
  This means that $R^{-1} = S$, as it has the same domain and has the same defining equation. Therefore, as $R^{-1}$ is closed, $S$ is closed, which makes it a closed extension of $T$. By definition, this means that $T$ is closable.
\end{proof}
