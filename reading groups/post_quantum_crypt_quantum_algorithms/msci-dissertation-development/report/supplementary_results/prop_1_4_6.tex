\subsection{Proof of Proposition \eqref{lbl_prop_spectrum_and_adjoint}}\label{proof_lbl_prop_spectrum_and_adjoint}

\begin{proposition}[{\cite[Proposition 2.7]{konrad}}]
  Let $T$ be a closed operator in $\HS$. Then,
    \begin{enumerate}[label = (\alph*)]
      \item $\rho(T)$ is the set of complex numbers such that $T - \lambda I \colon \dom{T} \to \HS$ is bijective.
      \item If $T$ is densely-defined, then
      \begin{enumerate}[label = (\roman*)]
        \item $\lambda \in \rho(T)$ if and only if $\conjugate{\lambda} \in \rho(T^*)$.
        \item $\lambda \in \sigma(T)$  if and only if $\conjugate{\lambda} \in \sigma(T^*)$.
      \end{enumerate}
  \end{enumerate}
\end{proposition}
\begin{proof}
  We recreate this proof from the proof of {\cite[Proposition 2.7]{konrad}}. For (a), if $\lambda$ is a regular point of $T$ then the existence of $(T - \lambda I)^{-1}$ with $\dom{T - \lambda I} = \HS$ is equivalent to $T - \lambda I$ being a bijective operator. This means that we just need to show that $T - \lambda I$ is a bounded operator.

  Now, as $T$ is a closed operator, by Corollary \eqref{lbl_closed_means_resolvent_function_closed} we must have that $T  - \lambda I$ is also closed. By Theorem \eqref{lbl_thrm_densely_defined_and_closability}(e), we then must have that $(T - \lambda I)^{-1}$ is closed. As $(T - \lambda I)^{-1}$ is closed and defined on the entirety of $\HS$, by the closed graph theorem, Theoorem \eqref{lbl_thrm_closed_graph}, this is equivalent to it being bounded. Therefore, $T - \lambda I$ has a bounded inverse if and only if $T - \lambda I$ is bijective, meaning that $\lambda$ is a regular point of $T$ is equivalent to the bijectivity of $T - \lambda I$.

  \medskip

  For part (b), first suppose that $T$ is densely-defined and that $\lambda$ is in the resolvent set of $T$. By Theorem \eqref{lbl_thrm_densely_defined_and_closability} (c), we must have that $(T - \lambda I)^* = T^* - \conjugate{\lambda}I$ is invertible as $T - \lambda I$ is, and that $\big( (T - \lambda I)^* \big)^{-1} = \big( (T - \lambda I)^{-1} \big)^*$. Now, as $\lambda$ is in the resolvent set of $T$, we have that $(T - \lambda I)^{-1}$ is bounded and defined on the entirety of $\HS$. This must mean that its adjoint is bounded and defined on the entirety. Therefore, as
  \begin{equation*}
    \big( (T - \lambda I)^{-1} \big)^*
    =
    (T^* - \conjugate{\lambda} I)^{-1},
  \end{equation*}
  we have that $T^* - \conjugate{\lambda} I$ has a bounded inverse defined on the entirety of $\HS$. By definition, this must mean that $\conjugate{\lambda}$ is in the resolvent set of $T^*$.

  Now suppose that $\lambda$ is some complex number such that $\conjugate{\lambda}$ is in the resolvent set of $T^*$. By following our above method, we will see that this means that $\lambda$ is in the resolvent set of $T$. Indeed, if $\conjugate{\lambda} \in \rho(T^*)$, then we have that $(T^* - \conjugate{\lambda} I)$ is invertible with its domain being the entirety of $\HS$; therefore, $(T^* - \conjugate{\lambda} I)^* = (T^*)^* - \lambda I$ is invertible  and is also defined on the entirety of $\HS$ by Theorem \eqref{lbl_thrm_densely_defined_and_closability}(c).  Now, as $T$ is closed, by Theorem \eqref{lbl_thrm_densely_defined_and_closability} (b) this is equivalent to $(T^*)^* = T$. Therefore, we have that $T - \lambda I$ must be an invertible operator defined on the entirety of $\HS$, which means that $\lambda$ is in the resolvent set of $T$. By putting this together, we therefore have that $\lambda$ is in the resolvent set of $T$ if and only if $\conjugate{\lambda}$ is in the resolvent set of $T^*$.

  \medskip

  Now, $\lambda$ being in the resolvent set of $T$ if and only if $\conjugate{\lambda}$ being in the resolvent set of $T^*$ means that $\lambda$ is not in the resolvent set of $T$ if and only if $\conjugate{\lambda}$ is not in the resolvent set of $T^*$. As the spectrum of an operator is the entirety of $\C$ with the resolvent set of the operator taken out, it follows that $\lambda$ is in the spectrum of $T$ if and only if $\conjugate{\lambda}$ is in the rspectrum of $T^*$.
\end{proof}
