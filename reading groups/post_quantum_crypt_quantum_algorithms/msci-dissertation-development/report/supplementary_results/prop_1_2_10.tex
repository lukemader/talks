\subsection{Proof of Proposition \eqref{lbl_prop_1_2_10}}\label{proof_lbl_prop_1_2_10}

\begin{proposition}[{\cite[Proposition 1.5]{konrad}}]
  Let $T$ be an operator in $\HS$. Then, the following are equivalent:
  \begin{enumerate}[label = (\alph*)]
    \item $T$ is closable.
    \item If $(x_n)_{n \in \N}$ is a sequence of vectors in $\dom{T}$ such that they converge to 0 and $(Tx_n)_{n \in \N}$ converges to some vector in $\HS$, then $(Tx_n)_{n \in  \N}$ converges to 0 too.
    \item $\overline{\G(T)}$ is the graph of a linear operator.
  \end{enumerate}
\end{proposition}
\begin{proof}
    We recreate this proof from the proof of {\cite[Proposition 1.5]{konrad}}, and start by showing that (a) implies (c). Suppose that $T$ is a closable operator, and $S$ is a closed extension of $T$ in $\HS$. Then, $\G(T) \subset \G(S)$ by Proposition \eqref{lbl_prop_extension_iff_graphs_subset}. As taking the closure of sets preserves subsets, we have that $\overline{\G(T)} \subset \overline{\G(S)}$. We also have that $\overline{\G(S)} = \G(S)$ as $S$ is closed; therefore, $\overline{\G(T)} \subset \G(S)$. Now, by  Lemma \eqref{lbl_lemma_subspace_conditions_to_be_graph}, as $\G(S)$ is the graph of an operator, we can only have that $(0, x) \in \G(S)$ if $x = 0$. As $\overline{\G(T)}$ is a subset of $\G(S)$, we therefore also have that $(0, x) \in \overline{\G(T)}$ when $x = 0$. Therefore, by Lemma \eqref{lbl_lemma_subspace_conditions_to_be_graph}, $\overline{\G(T)}$ must be the graph of some operator.

    \medskip

    We now  show that (c) implies (a). Suppose that there is some operator in $S$ such that $\overline{\G(T)}$ is the graph of $S$. By the properties of the closure of a set, we must also have that $\G(T) \subset \overline{\G(T)}$, so $\G(T) \subset \G(S)$, which means that there is some operator $T \subset S$ in $\HS$. As $\G(S)$ is clearly a closed set, this means that $T$ has a closed extension, and is therefore closable by definition.

    \medskip

    We have shown that (a) is equivalent to (c), so all that is left to show is that (b) is equivalent to (c). Suppose that (b) holds, which means that if $(0, x) \in \overline{\G(T)}$, we have that $Tx = 0$. The closure of a subspace is also a subspace (this is easily verified, so we omit the details here; for details, please see {\cite[Proposition 7.10]{muscat}}), so $\overline{\G(T)}$ is a subspace. This gives us that $\overline{\G(T)}$ is the graph of an operator by Lemma \eqref{lbl_lemma_subspace_conditions_to_be_graph}. If (c) holds, then $\overline{\G(T)}$ is the graph of an operator, so $(0, x) \in \overline{\G(T)}$ must imply that $x = 0$ by Lemma \eqref{lbl_lemma_subspace_conditions_to_be_graph}. This means that if $(x_n)_{n \in \N}$ is a sequence in $\dom{T}$ converging to 0  such that $(Tx_n)_{n \in \N}$ converges in $\HS$, then as $(x_n, Tx_n) = (0, \lim_{n \to \infty} Tx_n)$ is in $\overline{\G(T)}$, we have that $\lim_{n \to \infty}Tx_n = 0$, showing that (c) implies (b).
\end{proof}
