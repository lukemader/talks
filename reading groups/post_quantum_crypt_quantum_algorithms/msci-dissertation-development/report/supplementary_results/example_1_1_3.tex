\subsection{Proof of Example \eqref{lbl_example_multiplication_operator_dense_not_bounded}}\label{proof_lbl_example_multiplication_operator_dense_not_bounded}

\begin{example}[{\cite[Lemma 10.7-1]{kreyszig}}]
  Let $x$ be real-valued. We define the {\emph{multiplication by $x$ operator}} on $L^2(\R)$, $M_x\colon \dom{M_x} \to L^2(\R)$, by
  \begin{align*}
    \dom{M_x} &\coloneqq  \set{f \in L^2(\R) \colon xf \in L^2(\R)}, \\
    f(x) &\mapsto xf(x).
  \end{align*}
  Then, $M_x$ is a densely-defined operator which is not bounded.
\end{example}
\begin{proof}
  We  first show that $\dom{M_x}$ is a subspace of $L^2(\R)$, which just consists of a quick application of the subset test and the definition of $\dom{M_x}$. Firstly, we have that the zero function $0$ is in $\dom{M_x}$, as the zero function is in $L^2(\R)$ and as
  \begin{align*}
    \int_{\R} x 0(x) \,\mathrm{d}x
    &=
    \int_{\R} 0 \,\mathrm{d}x
    =
    0,
  \end{align*}
  so $x0(x)$ is also in $L^2(\R)$. Now, suppose that $f$ and $g$ are any functions in $\dom{M_x}$. Then, $f$, $g$, $xf(x)$, and $xg(x)$ are all elements of $L^2(\R)$. As $L^2(\R)$ is a vector space, it is closed under vector addition, which means that $f+g$ and $xf(x) + xg(x) = x(f(x) + g(x))$ are both elements of $L^2(\R)$. Therefore, $f+g$ must be in $\dom{M_x}$. Finally, for any complex number $\alpha$, we have that $\alpha f$ and $\alpha xf(x)$ are in $L^2(\R)$ as $f$ and $xf(x)$ both are and $L^2(\R)$ is a vector space and is therefore closed under scalar multiplication. This means that $\alpha f$ is in $\dom{M_x}$. Therefore, by the subspace test, $\dom{M_x}$ is a subspace.

  \medskip

  We now show that $\dom{M_x}$ is densely-defined. Let $x$ be real-valued and let $g$ be some arbitrary function in $L^2(\R)$. We first observe that
  % \begin{equation*}
  %   \abs{\frac{g}{abs^2 + 1}} \leq \abs{g}
  % \end{equation*}
  % and that
  \begin{equation*}
    \abs{\frac{x}{x^2 + 1}} \leq 1.
  \end{equation*}
  By taking the square of this inequality, we see that
  \begin{align*}
    \abs{\frac{xg(x)}{x^2 + 1}}^2
    &=
    \abs{\frac{x}{x^2 + 1}}^2 \abs{g(x)}^2
    \leq
    \abs{g(x)}^2.
  \end{align*}
  As $g$ is square-integrable, this means that
  \begin{align*}
    \int_{\R} \abs{\frac{xg(x)}{\abs{x}^2 + 1}} \, \mathrm{d}x
    &\leq
    \int_{\R} \abs{g(x)}^2 \, \mathrm{d}x
    <
    \infty.
  \end{align*}
  This therefore means that $\frac{g}{x^2 + 1}$ is in the domain of $M_x$.

  \medskip

  As $g$ was arbitrary, suppose that it actually lies in $\dom{M_x}^\perp$. This then means that
  \begin{align*}
    0
    &=
    \ip{\frac{g}{x^2 + 1}, g}
    =
    \int_{\R} \frac{\abs{g(x)}^2}{x^2 + 1} \, \mathrm{d}x,
  \end{align*}
  which is true if and only if $\abs{g}^2$ is 0 almost everywhere, which is true if and only if $g$ is 0 almost everywhere. As $g$ was an arbitrary function in $\dom{M_x}^\perp$, this means that
  \begin{equation*}
    \dom{M_x}^\perp = \set{0},
  \end{equation*}
  which is true if and only if $\dom{M_x}$ is dense in $L^2(\R)$. Therefore, $M_x$ is densely defined.

  \medskip

  We now follow the method in the proof of {\cite[Lemma 10.7-1]{kreyszig}} to show that $M_x$ is not bounded. For every natural number $n$, define the function
  \begin{equation*}
    f_n(x)
    \coloneqq
    \begin{cases}
      1 & \text{if $n \leq x < n+1$,}\\
      0 & \text{otherwise.}
    \end{cases}
  \end{equation*}
  We see that each $f_n$ is an element of $L^2(\R)$, as
  \begin{align*}
    \int_{\R} \abs{f_n (x)}^2  \, \mathrm{d}x
    &=
    \int_{n}^{n+1} 1^2 \,\mathrm{d}x
    =
    1.
  \end{align*}
  We also see that $x f_n(x)$ is in $L^2(\R)$ for every natural number $n$, as
  \begin{align*}
    \int_{\R}  \abs{x f_n(x)}^2 \, \mathrm{d}x
    &=
    \int_{n}^{n+1} x^2 \,  \mathrm{d}x
    =
    n^2 + n + \frac{1}{3}
    <
    \infty.
  \end{align*}
  This means that each function $x f_n(x)$ is in the domain of $M_x$. Now, we notice that for every $n$, we have that
  \begin{align*}
    \norm{M_x f_n}^2
    &=
    \int_{\R}  \abs{x f_n(x)}^2 \, \mathrm{d}x
    >
    n^2.
  \end{align*}
  Therefore, we have that $\norm{M_x f_n} > n$ for every natural number $n$. This means that $\norm{M_x f_n} \to \infty$ as $n \to \infty$, meaning that $M_x$ is not a bounded operator.
\end{proof}
