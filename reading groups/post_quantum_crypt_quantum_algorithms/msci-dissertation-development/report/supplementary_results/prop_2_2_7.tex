\subsection{Proof of Proposition \eqref{lbl_cannonical_commutation_relations}}\label{proof_lbl_cannonical_commutation_relations}

\begin{proposition}[{\cite[Proposition 3.25]{Hall2013}}]
  For a particle moving in $\R^n$, its position and momentum operators when defined on $\mathcal{S}(\R^n)$ satisfy
  \begin{align*}
    \frac{1}{i\hbar}[X_j, X_k] = 0, \quad
    \frac{1}{i\hbar}[P_j, P_k] = 0, \quad \text{and}\,\,
    \frac{1}{i\hbar}[X_j, P_k] = \delta_{j,k} I
  \end{align*}
  for any $1 \leq j \leq n$ and $1 \leq k \leq n$.
\end{proposition}
\begin{proof}
  For the position operators $X_j$ and $X_k$, we easily see that
  \begin{align*}
    X_j X_k \psi({x})
    &=
    x_j x_k \psi({x}) \\
    &=
    x_k x_j \psi({x}) \\
    &= X_k X_j \psi({x}),
  \end{align*}
  giving us that $[X_j, X_k] = 0$, which the first statement follows from. Similarly, remembering that second order partial differentials have symmetry $\frac{\partial^2}{\partial x \partial y } = \frac{\partial^2}{\partial y \partial x}$, we see that for the momentum operators $P_j$ and $P_k$,
  \begin{align*}
    P_j P_k \psi({x})
    &=
    -\hbar^2 \frac{\partial^2 \psi({x})}{\partial x \partial y } \\
    &=
    -\hbar^2 \frac{\partial^2 \psi({x})}{\partial y \partial x } \\
    &=
    P_k P_j \psi({x}),
  \end{align*}
  giving us that $[P_j, P_k] = 0$, which the second statement follows from. The final statement is proven for the single dimensional case in {\cite[Proposition 3.8]{Hall2013}}, which is easily adaptable for the $n$-dimensional case. We see that
  \begin{align*}
    P_k X_j \psi({x})
    &=
    -i \hbar \frac{\partial}{\partial x_k} \big( x_j \psi({x}) \big) \\
    &=
    -i \hbar \left( \frac{\partial x_j}{\partial x_k}\psi({x}) + x_j \frac{\partial \psi({x})}{\partial x_k} \right) \\
    &=
    \begin{cases}
      X_k P_j \psi({x}), & \text{if} \, j \neq k \\
      -i \hbar \psi({x}) + X_k P_j \psi({x}), & \text{if} \, j = k
    \end{cases},
  \end{align*}
  therefore giving us that $[X_j, P_k] = i\hbar \delta_{j,k} I$, giving us our result by re-arranging.
\end{proof}
