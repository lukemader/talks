\subsection{Proof of Corollary \eqref{lbl_corollary_sa_op_and_regular_points}}\label{proof_lbl_corollary_sa_op_and_regular_points}

\begin{corollary}[{\cite[Proposition 3.10]{konrad}}]
  Let $T$ be a self-adjoint operator in $\HS$. Then, the following are equivalent:
  \begin{enumerate}[label = (\alph*)]
    \item $\lambda$ is in the resolvent set of $T$.
    \item $\lambda$ is a regular point of $T$.
    \item $\range{T - \lambda I} = \HS$ and $d_\lambda (T) = 0$.
  \end{enumerate}
\end{corollary}
\begin{proof}
  We recreate this proof from the proof of {\cite[Proposition 3.10]{konrad}}. We notice that the implication of (b) from (a) has been done already for us by Proposition \eqref{lbl_resolvent_and_regular_point_relation}, which states that
  \begin{equation*}
    \rho(T) = \set{\lambda \in \pi(T) \colon d_\lambda(T) = 0},
  \end{equation*}
  so every element of the spectrum is necessarily a regular point of $T$.

  \medskip

  We now show that (b) implies (c). First, suppose that $\lambda$ is a regular point of $T$ and that it is not a real number. Proposition \eqref{lbl_sa_operator_real_spectrum} tells us that the spectrum of $T$ must be contained in the real numbers as $T$ is self-adjoint; therefore $\lambda$ must be in the resolvent set of $T$. By Proposition \eqref{lbl_prop_spectrum_and_adjoint}(b), this is true if and only if $\conjugate{\lambda}$ is in the resolvent set of $T^*$. As $T$ is self-adjoint, this means that $\conjugate{\lambda}$ is in the resolvent set of $T$ too. Therefore, $T - \conjugate{\lambda}I$ has a bounded inverse; specifically, it must be injective. Therefore, we have that
  \begin{align*}
    \set{0}
    &=
    \ker{T - \conjugate{\lambda} I} \\
    &=
    \ker{T^* - \conjugate{\lambda} I} & \text{as $T$ is self-adjoint} \\
    &=
    \ker{ (T - \lambda I)^*} & \text{by Proposition \eqref{lbl_prop_properties_of_adjoints}(f)} \\
    &=
    \left(\range{T - \lambda I}\right)^\perp & \text{by Proposition \eqref{lbl_prop_properties_of_adjoints}(a)}.
  \end{align*}

  Now suppose that $\lambda$ is actually real. As $\lambda$ is a regular point, there exists some real constant $c_\lambda > 0$ with
  \begin{equation*}
    c_\lambda \norm{x}
    \leq
    \norm{(T - \lambda I)x}
  \end{equation*}
  for all $x$ in the domain of $T$. This inequality shows us that $T - \lambda I$ is injective, as because $c_\lambda$ is greater than 0, $\norm{T - \lambda I} = 0$ if and only if $\norm{x} = 0$ which, by positive-semidefiniteness of the norm, is true if and only if $x = 0$. Therefore, \[\ker{T - \lambda I} = \set{0}.\] This gives us, similarly to before, that
  \begin{align*}
    \set{0}
    &=
    \ker{T - \lambda I} \\
    &=
    \ker{T^* - \conjugate{\lambda}I} & \text{as $T$ is self-adjoint and $\lambda$ is real} \\
    &=
    \ker{(T - \lambda I)^*} & \text{by Proposition \eqref{lbl_prop_properties_of_adjoints}(f)} \\
    &=
    \left(\range{T - \lambda I}\right)^\perp & \text{by Proposition \eqref{lbl_prop_properties_of_adjoints}(a)}.
  \end{align*}

  Therefore, if $\lambda$ is a regular point of $T$, we have that $\left(\range{T  - \lambda I}\right)^\perp = \set{0}$. This is true if and only if $\range{T - \lambda I}$ is dense. By Proposition \eqref{lbl_prop_adjoint_closed}, the adjoint of an operator is always closed, so as $T$ is self-adjoint it must be closed. Proposition \eqref{lbl_prop_properties_of_regular_points}(d) tells us that as $T$ is closed and $\lambda$ is a regular point for $T$, we have that $\range{T - \lambda I}$ is a closed subspace. As $\range{T - \lambda I}$ is closed and dense, we must have that $\range{T - \lambda I} = \HS$, as the closure of a dense subset is always the space it lives in. As $\left(\range{T - \lambda I}\right)^\perp = \set{0}$, we also have that
  \begin{align*}
      d_\lambda(T)
      &=
      \text{dim} \big( \left(\range{T - \lambda I}\right)^\perp \big) \\
      &=
      \text{dim}\big(\set{0}\big) \\
      &=
      0,
  \end{align*}
  meaning that (b) implies (c).

  \medskip

  We now show that (c) implies (a). Suppose first that $\lambda$ is a non-real complex number. Like before, as $T$ is self-adjoint, its spectrum is contained entirely in the real numbers by Proposition \eqref{lbl_sa_operator_real_spectrum}. Therefore, $\lambda$ must be in the resolvent set of $T$, as it is not real.

  \medskip

  Now suppose that $\lambda$ is a real number such that $\range{T - \lambda I} = \HS$. Specifically, this means that $T - \lambda I$ is surjective. It also means that we have that
  \begin{align*}
    \set{0}
    &=
    \HS^\perp \\
    &=
    \left( \range{T - \lambda I} \right)^\perp \\
    &=
    \ker{(T - \lambda I)^*} & \text{by Proposition \eqref{lbl_prop_properties_of_adjoints}(a)} \\
    &=
    \ker{T^* - \conjugate{\lambda}I} & \text{by Proposition \eqref{lbl_prop_properties_of_adjoints}(f)} \\
    &=
    \ker{T - \lambda I} & \text{as $T$ is self-adjoint and $\lambda$ is real.}
  \end{align*}
  Therefore, we have that $T - \lambda I$ is bijective.  Proposition \eqref{lbl_prop_spectrum_and_adjoint}(a) shows that the resolvent set of $T$ is precisely the complex numbers $\mu$ which make $T - \mu I$ bijective; therefore, $\lambda$ must be in the resolvent set of $T$. This means that (c) implies (a).
\end{proof}
