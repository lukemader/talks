\subsection{Proof of Proposition \eqref{lbl_prop_2_3_12}}\label{proof_lbl_prop_2_3_12}

\begin{proposition}[{\cite[p.232]{Hall2013}}]
  The operators $a$ and $a^*$ when defined on $\mathcal{S}(\R)$ can be expressed as
  \begin{align*}
    a   &= \frac{1}{\sqrt{2}}\left( \tilde{x} + \frac{d}{d\tilde{x}} \right), \\
    a^* &= \frac{1}{\sqrt{2}}\left( \tilde{x} - \frac{d}{d\tilde{x}} \right).
  \end{align*}
\end{proposition}
\begin{proof}
  This proof comes from the comments on {\cite[p.232]{Hall2013}}, which we display here with the gaps filled in. Notice that as $x = D \tilde{x}$, we have that
  \begin{equation*}
    \frac{d}{dx} = \sqrt{ \frac{m \omega}{\hbar} } \frac{d}{d \tilde{x}}.
  \end{equation*}
  Using these two bits of information, we see that
  \begin{align*}
    \frac{1}{\sqrt{2 \hbar m \omega}} \left( m \omega X \pm i P  \right)
    &= \frac{1}{\sqrt{2}} \left(  \frac{m \omega}{\sqrt{\hbar m \omega}} \sqrt{\frac{\hbar}{m \omega}} \tilde{x} \pm \frac{-i^2 \hbar}{\sqrt{\hbar m \omega}} \sqrt{\frac{m \omega}{\hbar}} \frac{d}{d \tilde{x}}  \right) \\
    &= \frac{1}{\sqrt{2}} \left( \tilde{x} \pm \frac{d}{d \tilde{x}} \right),
  \end{align*}
  giving us our alternative formulae for $a$ (the $+iP$ case) and $a^*$ (the $-iP$ case).
\end{proof}
