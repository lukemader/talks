\subsection{Proof of Proposition \eqref{lbl_prop_position_momentum_esa_on_S_R}}\label{proof_lbl_prop_position_momentum_esa_on_S_R}

\begin{proposition}[{\cite[Proposition 5.23, 5.29]{moretti}}]
  For a particle in $\R^n$, let $X_j \colon \dom{X_j} \to L^2(\R^n)$ be the position operator in dimension $1 \leq j \leq n$ and let $P_j \colon \dom{P_j} \to L^2(\R^n)$ be the momentum operator in dimension $1 \leq j \leq n$, where our domains are defined by the standard domains for the position and momentum operators introduced in Remark \eqref{lbl_remark_n_dimensional_position_op_sa} and in Remark \eqref{lbl_standard_domain_momentum_sa}.

  \medskip

  The position and momentum operators are essentially self-adjoint when restricted to the domain $\mathcal{S}(\R^n)$.
\end{proposition}
\begin{proof}
  The proof of part (b) relies on properties of the Fourier transform, so we omit them; the interested reader can find the proof in the proof of {\cite[Proposition 5.29]{moretti}}. We note that {\cite[Proposition 5.29]{moretti}} presents the result as $\mathcal{S}(\R^n)$ being a {\emph{core}} for $P_j$,  which are defined in {\cite[Definition 5.20]{moretti}}. It is then shown in {\cite[Proposition 5.21]{moretti}} that this implies that $P_j$ is essentially self-adjoint on $\mathcal{S}(\R^n)$.

  \medskip

  We recreate the proof of part (a) from the proof of {\cite[Proposition 5.23]{moretti}}. We first notice that every function in $\mathcal{S}(\R^n)$ is a function in $\dom{X_j}$. This follows easily from the definition of $\mathcal{S}(\R^n)$; as
  \begin{equation*}
    \sup_{x \in \R^n}
    \abs{
          \prod_{j = 1}^{n} \big( x_j^{y_j}\big)
          \frac{
                \partial^{s_z} f(x)
          }{
                \partial x_1^{z_1} \partial x_2^{z_2} \cdots \partial x_n^{z_n}
          }
    }
    <
    \infty
  \end{equation*}
  for every $y, z \in \N_0^n$, this must be true for $x_j f(x)$, as all we are doing is increasing $y_j$ by one when we multiply by $x_j$.  As $\mathcal{S}(\R^n)$ and $\dom{X_j}$ are subspaces of $L^2(\R^n)$, and $\mathcal{S}(\R^n)$ is dense in $L^2(\R^n)$, it follows that $\mathcal{S}(\R^n)$  is a dense subspace of $\dom{X_j}$.

  \medskip

  Now, we know from Remark \eqref{lbl_remark_n_dimensional_position_op_sa} that $X_j$ is self-adjoint on $\dom{X_j}$. By Theorem \eqref{lbl_thrm_sa_criteria}, this means that on $\dom{X_j}$,
  \begin{equation*}
    \range{X_j \pm iI} = \HS.
  \end{equation*}
  Now, by the self-adjointness of $X_j$ and $I$ and by Proposition \eqref{lbl_prop_properties_of_adjoints}(f) and (a), we have that
  \begin{align*}
    \ker{X_j \pm iI}
    &=
    \ker{X_j^* \pm iI^*} \\
    &=
    \ker{\big(X_j \pm iI \big)^*} \\
    &=
    \left(\range{X_j \pm iI} \right)^\perp \\
    &=
    \HS^\perp \\
    &=
    \set{0}.
  \end{align*}
  If we let $X_j \restriction_{\mathcal{S}(\R^n)}$ denote the domain restriction of $X_j$ onto $\mathcal{S}(\R^n)$, then, as $X_j$ is self-adjoint, we must have that $\left( X_j \restriction_{\mathcal{S}(\R^n)}\right)^*$ agrees  with $X_j$ on $\mathcal{S}(\R^n)$. This means that we must also have that
  \begin{equation*}
    \ker{
          \left( X_j \restriction_{\mathcal{S}(\R^n)}\right)^* \pm iI
        }
      =
      \ker{X_j \pm iI} = \set{0}.
  \end{equation*}
  Therefore, by Theorem \eqref{lbl_thrm_esa_criteria}(c), we must have that $X_j \restriction{\mathcal{S}(\R^n)}$ is essentially self-adjoint.
\end{proof}
