In this section, we provide a brief summary of Lebesgue integration. Unfortunately, we do not have time to give this subject justice, so we instead aim to create a working knowledge of the topic. For full details and a proper construction of our results, please see {\cite[Chapters 1,2]{stein}} and {\cite[Appendix A]{teschl}}, whose relevant results display here.

\subsubsection{An introduction to measure theory}

We start by introducing the concept of a measure space.

\begin{definition}[{\cite[p.259-260]{teschl}}]
  Let $\Omega$ be any set. Then, a collection of subsets of $\Omega$, $\mathcal{F}$, is a {\emph{$\sigma$-algebra}} if
\begin{enumerate}[label=(\alph*)]
  \item $\Omega \in \mathcal{F}$.
  \item $\mathcal{F}$ is closed under complements; for all $A \in \mathcal{F}$, $A^\mathrm{C} \in \mathcal{F}$.
  \item $\mathcal{F}$ is closed under countable unions; for any countable collection of sets $(A_n)_{n \in \N}$ in $\mathcal{F}$, $\cup_{j=1}^{\infty}A_j \in \mathcal{F}$.
\end{enumerate}
If $\mathcal{F}$ is a $\sigma$-algebra for $\Omega$, we call $(\Omega, \mathcal{F})$ a {\emph{measurable space}}. Notice that $\Omega^{\mathrm{C}} = \emptyset \in \mathcal{F}$, and as $\mathcal{F}$ is closed under countable unions, by de Morgan's laws it is closed under countable intersections too.

\medskip

For a measurable space $(\Omega, \mathcal{F})$, a mapping $\mu: \mathcal{F} \to [0, \infty]$ is called a {\emph{measure}} if
\begin{enumerate}[label=(\alph*)]
  \item $\mu(\emptyset) = 0$.
  \item For any countable collection $(A_n)_{n \in \N}$ of sets in $\mathcal{F}$ which are pairwise disjoint, $A_j \cap A_k = \emptyset$ for $j \neq k$, then \[\mu\left(\bigcup_{j=1}^{\infty}A_j\right) = \sum_{j=1}^{\infty}\mu(A_j).\]
\end{enumerate}
The property (b) is referred to as {\emph{countable additivity}}. For a measurable space $(\Omega, \mathcal{F})$ and a measure $\mu: \mathcal{F} \to [0, \infty]$, we call $(\Omega, \mathcal{F}, \mu)$ a {\emph{measure space}}. We say that a set $E \in \mathcal{F}$ is {\emph{measurable}}.
\end{definition}
\begin{example}[{\cite[p.260]{teschl}}]
  An important example of a $\sigma$-algebra for a space $\Omega$ is the Borel sigma-algebra $\mathcal{B}(\Omega)$, which is the $\sigma$-algebra generated by all open sets. When $\Omega = \R^n$, we use the notation $\mathcal{B}_n \coloneqq \mathcal{B}(\R^n)$. We will always assume that $\R^n$ is equipped with the Borel $\sigma$-algebra.
\end{example}

If $A = \cup_{n \in \N}A_n$ for some a sequence of sets $(A_n)_{n \in \N}$, we write $A_n \nearrow A$ if $A_n \subset A_{n+1}$ and we write $A_n \searrow A$ if $A_{n+1} \subset A_n$.

\begin{proposition}[{\cite[Theorem A.1]{teschl}}]
  For any measure space $(\Omega, \mathcal{F}, \mu)$, the measure $\mu$ satisfies the following:
  \begin{enumerate}[label=(\alph*)]
    \item For $A$ and $B$ being sets in $\mathcal{F}$ with $A \subset B$, then $\mu(A) \leq \mu(B)$.
    \item If $A_n \nearrow A=\cup_{j=1}^{\infty}A_j$, then $\mu(A_n) \to \mu(A)$ as $n \to \infty$.
    \item If $A_n \searrow A=\cup_{j=1}^{\infty}A_j$ and $\mu(A_1) < \infty$, then $\mu(A_n) \to \mu(A)$ as $n \to \infty$.
  \end{enumerate}
\end{proposition}
\begin{proof}
  As the proof of {\cite[Theorem A.1]{teschl}} points out, this is a relatively simple proof which relies on a few tricks. We recreate the method described in the proof of {\cite[Theorem A.1]{teschl}} here. For part (a), suppose that $A \subset B$, and let $C = B \backslash A$. Then, $A \cap C = \empty$ and $A \cup C = B$. By the countable additivity of the measure $\mu$, we then have that
  \begin{equation*}
    \mu(A \cup C) = \mu(A) + \mu(C) = \mu(B).
  \end{equation*}
  As a measure must have non-negative values, this then gives us our desired inequality of
  \begin{equation*}
    \mu(A) \leq \mu(B).
  \end{equation*}
  For part (b), suppose that $A_n \nearrow A = \cup_{j=1}^{\infty}A_j$. Then, if we set $\tilde{A}_n = A_n \backslash A_{n-1}$, we have that
  \begin{equation*}
    A_{n-1} \cap \tilde{A}_n = \empty,\,\,\text{and}\,\,A_{n-1} \cup\tilde{A}_n = A_n.
  \end{equation*}
  As $\cup_{j=2}^{n}\tilde{A}_j = A_n$, by the countable additivity of $\mu$ we have that
  \begin{align*}
    \mu(A)
    &=
    \mu\left( \bigcup_{j=2}^{\infty}\tilde{A}_j \right) \\
    &=
    \lim_{n \to \infty} \sum_{j=2}^{n} \mu(\tilde{A}_j) \\
    &=
    \lim_{n \to \infty} \mu\left( \bigcup_{j=2}^{n}\tilde{A}_j \right) \\
    &=
    \lim_{n \to \infty} \mu(A_n)
  \end{align*}
  as required. Part (c) works in a similar way to part (b). Let's assume that $A_n \searrow A=\cup_{j=1}^{\infty}A_j$ and $\mu(A_1) < \infty$. Then, if we let $\tilde{A}_n = A_n \backslash A_{n+1}$, we have that
  \begin{equation*}
    A_1 = A \cup \bigcup_{j=1}^{\infty}\tilde{A}_j.
  \end{equation*}
  By the countable additivity of $\mu$ and the definition of $\tilde{A}_n$, we therefore have that
  \begin{align*}
    \mu(A_1)
    &=
    \mu(A) + \sum_{j=1}^{\infty} \mu(\tilde{A}_j) \\
    &=
    \mu(A) + \sum_{j=1}^{\infty} \big( \mu(A_j) - \mu{A_{j+1}} \big) \\
    &=
    \mu(A) + \lim_{n \to \infty} \sum_{j=1}^{n-1} \big( \mu(A_j) - \mu{A_{j+1}} \big) \\
    &= \mu(A) + \mu(A_1) - \lim_{n \to \infty}\mu(A_n).
  \end{align*}
  As $\mu(A_1)$ is finite by assumption, by re-arranging, we get our require result of
  \begin{equation*}
    \lim_{n \to \infty}\mu(A_n) = \mu(A).
  \end{equation*}
\end{proof}

\begin{definition}[{\cite[p.269]{teschl}}]
  Let $(\Omega_1, \mathcal{F}_1)$ and $(\Omega_2, \mathcal{F}_2)$ be two measurable spaces. A function $f: \Omega_1 \to \Omega_2$ is called {\emph{measurable}} if for all $E \in \mathcal{F}_2$, the pre-image of $E$, $f^{-1}(E)$, is in $\mathcal{F}_1$.
\end{definition}
\begin{remark}
  As {\cite[p.28]{stein}} mentions, on $\R^n$ equipped with the Borel $\sigma$-algebra, we have that $f$ is measurable if
  \begin{equation*}
    f^{-1}\big( [-\infty, a) \big)
  \end{equation*}
  is measurable for all real numbers $a$. As this is true for all $a$, this is equivalent to saying that
  \begin{equation*}
    f^{-1}\big( [-\infty, a] \big)
  \end{equation*}
  is measurable for all real numbers $a$, As a $\sigma$-algebra contains all of the complements of its sets by definition, this is also equivalent to saying that the set
  \begin{equation*}
    \set{f > a} \coloneqq f^{-1}\big( (a, \infty] \big)
  \end{equation*}
  is measurable.
\end{remark}

The reason that we care about measure spaces and integrable functions is that we can construct a form of integration for them. We construct this integral in stages; we start by definining it for a type of measurable function called a {\emph{simple measure function}} under the condition that they're non-negative, expand the definition to non-negative measurable functions, and then use this to define the integral for all negative functions.

\begin{definition}[{\cite[p.270]{teschl}}]
  Let $(\Omega, \mathcal{F})$ be a measurable space. A measurable function $f: (\Omega, \mathcal{F}) \to (\R^n, \mathcal{B}_n)$ is called {\emph{simple}} if there exist a finite collection of disjoint sets $A_1, \cdots, A_n \in \mathcal{F}$ and a finite collection of complex numbers $\alpha_1, \cdots, \alpha_n$ such that
  \begin{equation*}
    f(x) = \sum_{j=1}^{n} \alpha_j 1_{A_j}(x),
  \end{equation*}
  where $1_{A_j}$ is the indicator function for $A_j$.

  \medskip

  If $\mu$ is a measure on $(\Omega, \mathcal{F})$, then we define the integral of a simple function $f$ with respect to $\mu$ over a set $A \in \mathcal{F}$ as
  \begin{equation*}
    \int_{A} f \,\mathrm{d}\mu \coloneqq \sum_{j=1}^{n} \alpha_j \mu(A_j \cap A).
  \end{equation*}
\end{definition}
For the rest of this chapter, if we are integrating over the entirety of $\Omega$, we will adopt the convention of dropping this from the bounds of the integral.

\medskip

The following results show that this integral behaves in the way that we are used to.

\begin{proposition}[{\cite[Lemma 1.13]{teschl}}]\label{lbl_prop_simple_functions_integral_properties}
  For a measure space $(\Omega, \mathcal{F}, \mu)$, the integral with respect to $\mu$ of simple functions $f$ and $g$ satisfy:
  \begin{enumerate}[label = (\alph*)]
    \item $\int_A f \,\mathrm{d}\mu = \int_{\Omega} f 1_A \,\mathrm{d}\mu$.
    \item $\int_{\bigcup_{n \in \N} A_n} f \,\mathrm{d}\mu = \sum_{n \in \N} \int_{A_j}f \,\mathrm{d}\mu$.
    \item For all $\alpha \geq 0$, $\int_A \alpha f \,\mathrm{d}\mu = \alpha  \int_A f \,\mathrm{d}\mu$.
    \item $\int_A f + g \,\mathrm{d}\mu = \int_A f \,\mathrm{d}\mu + \int_A g \,\mathrm{d}\mu$.
    \item If $A \subset B$, then $\int_A f \,\mathrm{d}\mu \leq \int_B f \,\mathrm{d}\mu$.
    \item If $f \leq g$, then $\int_A f \,\mathrm{d}\mu \leq \int_A g \,\mathrm{d}\mu$.
  \end{enumerate}
\end{proposition}
\begin{proof}
  This proof is relatively simple, and we therefore omit it. For details of the proof, please see the proof of {\cite[Lemma 1.13]{teschl}} or {\cite[Chapter 2, Proposition 1.1]{stein}}.
\end{proof}

\begin{proposition}[{\cite[Property 5, p.29]{stein}}]\label{lbl_prop_addition_measurable_functions_measurable}
  If $f$ and $g$ are two measurable functions on $\R^n$ equipped with the Borel $\sigma$-algebra, then
  \begin{enumerate}[label=(\alph*)]
    \item For any integer $n \geq 1$, $f^n$ is measurable.
    \item If $f$ and $g$ are finite-valued, then $f+g$ and the product $fg$ are both measurable.
  \end{enumerate}
\end{proposition}
\begin{proof}
  This proof is relatively simple, and we recreate it from the proof of {\cite[Property 5, p.29]{stein}}. For part (a), first assume that $n$ is odd and let $a$ be any real number. As $n$ is odd, we only have one possible sign to the root to $f$, so we then have that
  \begin{equation*}
    \set{f^n > a} = \set{f > a^{\frac{1}{n}}},
  \end{equation*}
  which is measurable as $f$ is. If $n$ is even, we then have two possible signs to the root of $f$, so we have that
  \begin{equation*}
    \set{f^n > a} = \set{f > a^\frac{1}{n}} \cup \set{f < -a^{\frac{1}{n}}},
  \end{equation*}
  which are both measurable sets as $f$ is measurable. As the union of two measurable sets is another measurable set by the definition of the $\sigma$-algebra, we have that $f^n$ is measurable when $n$ is even. As $f^n$ is measurable for both even and odd $n$, it is therefore a measurable function for all integers $n \geq 1$.

  \medskip

  For part (b), suppose that $f$ and $g$ are both measurable functions. We then have notice the trick that
  \begin{equation*}
    \set{f + g > a}
    =
    \bigcup_{r \in \Q} \big( \set{f > a - r} \cup \set{g > r} \big).
  \end{equation*}
  These two sets in the union are both countable as $f$ and $g$ both are and $a - r$ is real for all real $a$ and rational $r$. As $\Q$ is countable and the countable union of measurable sets is measurable by the definition of a $\sigma$-algebra, we have that this set is measurable. Therefore, $f+g$ is both measurable.

  \medskip

  For the product $fg$, remember that we just showed that $f + g$ is measurable. This means that $f - g$ is also measurable, and we therefore have by part (a) that $(f+g)^2$ and $(f-g)^2$ is measurable. Therefore, we also have that $(f+g)^2 - (f - g)^2$ is measurable. We now notice that
  \begin{equation*}
    fg = \frac{1}{4} \big( (f+g)^2 - (f - g)^2  \big).
  \end{equation*}
  It is easily verifiable that the scalar multiple of a measurable function is measurable, as for all $a \in \R$ and some real constant $\alpha$, we have that
  \begin{equation*}
    \set{\alpha f > a } = \set{f > \frac{a}{\alpha}}.
  \end{equation*}
  This means that $fg$ must be measurable as
  \begin{equation*}
    \frac{1}{4} \big( (f+g)^2 - (f - g)^2  \big)
  \end{equation*}
  is.
\end{proof}

A fundamental concept in measure theory is the idea of functions being equal {\emph{almost everywhere}}.

\begin{definition}[{\cite[p.30]{stein}}]
  Let $f$ and $g$ be two functions defined on a shared measurable domain $A$. We say that $f$ and $g$ are {\emph{equal almost everywhere}} if
  \begin{equation*}
    \set{x \in A \,:\, f(x) \neq g(x)}
  \end{equation*}
  has a measure of 0.
\end{definition}

We now see that we can approximate measurable functions by simple functions.

\begin{theorem}[{\cite[Chapter 1, Theorem 4.2]{stein}}]\label{lbl_thrm_measurable_simple_approx}
  Let $f$ be a measurable function on $\R^n$ equipped with the Borel $\sigma$-algebra. Then, there exists a sequence $(\psi_n)_{n \in \N}$ of simple functions such that their absolute values are an increasing sequence and such that they pointwise converge to $f$.
\end{theorem}
\begin{proof}
  This proof is fairly simple when relying on the trick of writing our function $f$ as a decomposition of its positive and negative parts. We therefore omit it; for details, please see the proof of {\cite[Chapter 1, Theorem 4.2]{stein}}.
\end{proof}

We can now expand this result to approximate by step functions rather than simple functions. An issue with this development, however, is that our pointwise convergence in general can fail, but instead they converge almost everywhere.

\begin{theorem}[{\cite[Chapter 1, Theorem 4.3]{stein}}]
  Let $f$ be a measurable function on $\R^n$. There then exists a sequence of step functions $(\phi)_{n \in \N}$ that pointwise converge to $f(x)$ for almost every $x$.
\end{theorem}
\begin{proof}
  We omit this proof for time reasons, as it relies on properties of the symmetric difference of a measurable set with a union of cubes. For details, please see the proof of {\cite[Chapter 1, Theorem 4.3]{stein}}.
\end{proof}


We can now move on to extending our integral of a simple function to bounded functions which are supported on a set with finite measure. We do so by recreating {\cite[p.53-57]{stein}}.

\begin{definition}[{\cite[p.53]{stein}}]\label{lbl_def_support}
  For a measurable function $f$, the {\emph{support}} of $f$ is the set
  \begin{equation*}
    \text{supp}(f) \coloneqq \set{x \,: \, f(x) \neq 0}.
  \end{equation*}
  If $A$ is a subset of the domain of $f$, we say that $f$ is {\emph{supported on E}} if for all $x \not\in A$, $f(x) = 0)$.
\end{definition}

By Theorem \eqref{lbl_thrm_measurable_simple_approx} and the following lemma, we can now expand our integral to bounded functions supported on a set with finite measure.

\begin{lemma}[{\cite[Chapter 2, Lemma 1.2]{stein}}]\label{lbl_measurable__bounded_function_supported_finite_measure_approximation}
  Let $f$ be a function that is supported on some set $A$ with finite measure and that is bounded by some $M > 0$. Suppose that $(\psi_n)_{n \in \N}$ is a sequence of simple functions also supported by $A$, bounded by $M$, and converge to $f(x)$ for almost every $x$. Then,
  \begin{enumerate}[label = (\alph*)]
    \item $\lim_{n \to \infty} \int \psi_n \,\mathrm{d}\mu$ exists.
    \item $\lim_{n \to \infty} \int \psi_n \,\mathrm{d}\mu = 0$ if $f(x) = 0$ for almost every $x$.
  \end{enumerate}
\end{lemma}
\begin{proof}
  This proof is relatively simple, but relies on the famous result known as {\emph{Egorov's theorem}}, which we omit due to our time constraints. For details, please see {\cite[Chapter 2, Lemma 1.2]{stein}}.
\end{proof}

This allows us to define the integral of a bounded function supported on a set of finite measure as
\begin{equation*}
   \int f \,\mathrm{d}\mu =  \lim_{n \to \infty} \int \psi_n \,\mathrm{d}\mu.
\end{equation*}
{\cite[p.54]{stein}} shows that Lemma \eqref{lbl_measurable__bounded_function_supported_finite_measure_approximation} is well-defined; we omit the details here, as this is a simple verification by just assuming that there are two distinct sequences of simple functions which are bounded and supported on the sames set as $f$ and which approximate $f$, and then showing that the limit of the integrals of these two sequences is in fact equal.

\medskip

We also note that Proposition \eqref{lbl_prop_simple_functions_integral_properties} excluding (b) and (d) is also true for our definition of the integral for bounded functions supported on a set of finite measure. For details, please see {\cite[Chapter 2, Proposition 1.3]{stein}}.

\begin{theorem}[Bounded convergence theorem, {\cite[Chapter 2, Theorem 1.4]{stein}}]
  Let $(f_n)_{n \in \N}$ be a sequence of measurable functions which are supported on a measurable set $A$ with finite measure, are all bounded by some $M > 0$, and converge to some function $f$ for almost every $x$ as $n \to \infty$. Then, $f$ is also measurable, bounded, and supported on $A$ for almost every $x$, and
  \begin{equation*}
    \int \abs{f_n - f} \to 0 \qquad \text{as $n \to \infty$}.
  \end{equation*}
\end{theorem}
\begin{proof}
  Where this result is extremely important, its proof again relies on Egorov's theorem, and we therefore omit it; for details, please see the proof of {\cite[Chapter 2, Theorem 1.4]{stein}}.
\end{proof}

We can now expand our integral definition to measurable functions that do not have to be bounded and supported on a set with finite measure, but instead are just non-negative. We do this by recreating {\cite[p.58-p.64]{stein}} To do so, we will need to allow our functions to take on the value $+\infty$.

\medskip

For a measurable non-negative function $f$ and for all bounded measurable functions $g$ which are supported on a set with finite measure and such that $0 \leq g \leq f$, we define the integral of $f$ as
\begin{equation*}
  \int f(x) \,\mathrm{d}\mu = \sup_{0 g \leq f} \int g(x) d \mu.
\end{equation*}
If this integral is finite, we say that $f$ is integrable. Just like before, this integral satisfies all of the properties we'd want it to have of Proposition \eqref{lbl_prop_simple_functions_integral_properties}, apart from maybe (b) and (d). For details, please see {\cite[Chapter 2, Proposition 1.6]{stein}}.

\medskip

We can now complete our construction and define the integral for any real-valued function f on $\R^n$. We do this by recreating {\cite[p.64-67]{stein}}. We say that a real-valued measurable function $f$ on $\R^n$ is integrable if
\begin{equation*}
  \int \abs{f} \,\mathrm{d}\mu < \infty.
\end{equation*}
We denote the space of all measurable functions on $\R^n$ by $L^{1}(\R^n, \mu)$. In order to give a value to this integral, we will need the following result.

A useful result we will need for the construction of more general integrals is the following.

\begin{proposition}[{\cite[Chapter 2, Proposition 1.6]{stein}}]\label{lbl_prop_measurable_function_bound_integrable}
  Let $f, g \colon \R^n \to \R$ be non-negative measurable functions. Then, if $g$ is integrable and $f$ satisfies $0 \leq f \leq g$, then $f$ is also integrable.
\end{proposition}
\begin{proof}
  By Proposition \eqref{lbl_prop_addition_measurable_functions_measurable}, we can define a measurable function $h = g - f$. As we assume that $f \leq g$, we must have that $h$ is a non-negative function. This then implies that
  \begin{equation*}
    \int h(x) \,\mathrm{d}\mu \leq 0
  \end{equation*}
  by the properties of the integral for a non-negative measurable function. By substituting our definition for $h$ and using the properties of the integral for non-negative measurable functions, we then see that
  \begin{equation*}
    \int f(x) \,\mathrm{d}\mu \leq   \int g(x) \,\mathrm{d}\mu.
  \end{equation*}
  As $g$ is integrable, the integral on the right is finite-valued; therefore, $f$ must be integrable too.
\end{proof}


To actually give a value to the integral of $f$, we define the following two non-negative functions
\begin{align*}
  f^{+}(x) &= \text{max}(f(x), 0),\,\text{and}\\
  f^{-}(x) &= \text{max}(-f(x), 0).
\end{align*}
As $f^{+} \leq \abs{f}$ and $f^{-} \leq \abs{f}$, they are both measurable by Proposition \eqref{lbl_prop_measurable_function_bound_integrable}, so we define the integral of $f$ as
\begin{equation*}
  \int f(x) \,\mathrm{d}\mu \coloneqq \int f^+(x) \,\mathrm{d}\mu - \int f^-(x) \,\mathrm{d}\mu.
\end{equation*}

This integral follows all of the properties the integral for non-negative measurable functions follows, which is provided in {\cite[Chapter 2, Proposition 1.6]{stein}}. Importantly for us, we have linearity, additivity of disjoint subsets of $\R^n$, monotonicity, and the implication that if the integral of a function is 0, then the function is the zero function almost everywhere.

\medskip

A very important theorem for us is the {\emph{dominated convergence theorem}}, which is the following result.

\begin{theorem}[Dominated Convergence Theorem, {\cite[Chapter 2, Theorem 1.13]{stein}}]\label{lbl_thrm_dominated_convergence}
  Let $(f_n)_{n \in \N}$ be a sequence of measurable functions which converge to some function $f(x)$ for almost every $x$, and let $g$ be some integrable measurable function such that $\abs{f_n(x)} \leq g(x)$ for all $n \in \N$. Then, $f_n$ and $f$ are both integrable, with
  \begin{equation*}
    \int \abs{f_n - f} \,\mathrm{d}\mu \to 0 \qquad \text{as $n \to \infty$.}
  \end{equation*}
\end{theorem}
\begin{proof}
  This proof is relatively quite simple, and so we omit it due to our time constraints; for details, please see the proof of {\cite[Chapter 2, Theorem 1.13]{stein}}.
\end{proof}

Finally, we can expand our integral to complex-valued functions on $\R^n$.  We do this by recreating {\cite[p.67-68]{stein}}. If $f$ is a complex-valued function, we can write it as
\begin{equation*}
  f(x) = \text{real}(f(x)) + i\,\text{im}(f(x)).
\end{equation*}
We see that $f$ is integrable if and only if its real and imaginary parts are integrable, as
\begin{align*}
  \abs{\text{real}(f(x))} &\leq \abs{f(x)}, \\
  \abs{\text{im}(f(x))} &\leq \abs{f(x)}, \quad \text{and} \\
  \abs{f(x)} &\leq \abs{\text{real}(f(x))} + \abs{\text{im}(f(x))} \leq 2\abs{f(x)}.
\end{align*}

We therefore can define our integral as
\begin{equation*}
  \int f(x) \,\mathrm{d}\mu \coloneqq \int \text{real}(f(x)) \,\mathrm{d}\mu + i \int \text{im}(f(x)) \,\mathrm{d}\mu.
\end{equation*}

\subsubsection{The Lebesgue measure}

In the previous section, we introduced the idea of a measure space and gave a brief overview of how we can integrate complex measurable functions. An important measure to us, particularly in our last chapter on quantum physics, will be the {\emph{Lebesgue measure}}. To develop this, we first define what we mean by a {\emph{closed cube}} in $\R^n$.

\begin{definition}[{\cite[p.3-4]{stein}}]
  Let $a_1, \cdots, a_n, b_1, \cdots, b_n$ be real numbers such that $b_1 - a_1 = \cdots = b_n - a_n$. A {\emph{closed cube}} in $\R^n$ is a set
  \begin{equation*}
    Q = \set{(x_1, \cdots, x_n) \in \R^n \,:\, a_j \leq x_j \leq b_n \,\, \text{for all}\,\, 1 \leq j \leq n}.
  \end{equation*}
  The {\emph{interior of the cube}} is defined to be the set
  \begin{equation*}
    \text{interior}(R) = \prod_{j=1}^{n}(a_j, b_j).
  \end{equation*}
  We define the {\emph{volume}} of a closed cube $Q$ to be the quantity
  \begin{align*}
    \abs{Q} = \prod_{j=1}^{n} b_j - a_j.
  \end{align*}
  We say that a union of cubes is {\emph{almost disjoint}} if the interiors of each cube is mutually disjoint.
\end{definition}

\begin{theorem}[{\cite[Chapter 1, Theorem 1.4]{stein}}]
  Every open subset of $\R^n$ can be written as a countable of almost disjoint cubes.
\end{theorem}
\begin{proof}
  This proof is simple but long, and relies on an algorithm that is best demonstrated by an image. We therefore omit it; for details, please see the proof of {\cite[Chapter 1, Theorem 1.4]{stein}}.
\end{proof}

We can now define the Lebesgue measure for $\R^n$.

\begin{definition}[{\cite[p.10, 16]{stein}}]
  We define the {\emph{Lebesgue measure}} as function $m: (\R^n, \mathcal{B}_m) \to [0, \infty]$ as the function
  \begin{equation*}
    m(A) = \inf \sum_{j=1}^{\infty} \abs{Q_j},
  \end{equation*}
  where the infinum of the sum is taken over all countable coverings of $A$ of closed cubes, $A \subset \cup_{j \in \N}Q_j$.
\end{definition}
From the definition, we see that every open set in $\R^n$ is Lebesgue-measurable as they are in the Borel algebra by definition. The Borel algebra also has the closed sets in $\R^n$, as a set is open if and only if its complement is closed and a $\sigma$-algebra contains all the complements of the sets inside of it. This also means that every closed interval is Lebesgue-measurable.

\medskip

{\cite[Chapter 2, Theorem 1.5]{stein}} proves that the Riemann integral on a closed interval $[a,b] \subset \R$ is equal to the Lebesgue integral on the same closed interval. We therefore adopt the common notation of writing $d x$ instead of $d m$ in our integrals; that is, the Lebesgue integral of some measurable function $f$ is denoted by

\begin{equation*}
  \int_{\R} f(x) d x \coloneqq \int_{\R} f(x) d m.
\end{equation*}
