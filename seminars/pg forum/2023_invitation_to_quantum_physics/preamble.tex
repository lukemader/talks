\usepackage[export]{adjustbox}
\usepackage{amsmath,
            amssymb,
            gensymb,
            mathtools
            }

\usepackage{textcomp} % i forget this package's purpose

% \usepackage{pgfplots}
% \pgfplotsset{width=10cm}
\linespread{1.8}


\usepackage{amsthm}

\newcommand{\set}[1]{\left\{#1\right\}}
\newcommand{\N}{\mathbb{N}}
\newcommand{\R}{\mathbb{R}}
\newcommand{\Z}{\mathbb{Z}}
\newcommand{\Q}{\mathbb{Q}}
\newcommand{\C}{\mathbb{C}}
\newcommand{\M}{\mathrm{M}}
\newcommand{\F}{\mathbb{F}}
\newcommand{\HS}{\mathcal{H}}

\newcommand{\eps}{\varepsilon}
\newcommand{\id}{\text{id}}
\newcommand{\dist}[1]{\text{dist}\left(#1\right)}
\newcommand{\ip}[1]{\left\langle #1\right\rangle}
\newcommand{\conjugate}[1]{\mkern 1.5mu\overline{\mkern-1.5mu#1\mkern-1.5mu}\mkern 1.5mu} 
\newcommand{\dom}[1]{\mathrm{Dom}\left(#1\right)}


\DeclarePairedDelimiter\abs{\lvert}{\rvert}%
\DeclarePairedDelimiter\norm{\lVert}{\rVert}%

% Swap the definition of \abs* and \norm*, so that \abs
% and \norm resizes the size of the brackets, and the
% starred version does not.
\makeatletter
\let\oldabs\abs
\def\abs{\@ifstar{\oldabs}{\oldabs*}}
%
\let\oldnorm\norm
\def\norm{\@ifstar{\oldnorm}{\oldnorm*}}
\makeatother


\newcommand{\op}[1]{\norm{#1}_\text{op}}



% bra-ket notation
% uses package mathtools
\DeclarePairedDelimiter\bra{\langle}{\rvert}
\DeclarePairedDelimiter\ket{\lvert}{\rangle}
\DeclarePairedDelimiterX\braket[2]{\langle}{\rangle}{#1 \vert #2}
\DeclarePairedDelimiterX\ketbra[2]{\delimsize\vert}{\delimsize\vert}{#1 \rangle\langle #2}





\newcommand*{\rvec}[1]{\left( #1\right)}
\newcommand*{\tr}[1]{\operatorname{tr}\left(#1\right)}
\newcommand*{\range}[1]{\operatorname{range}\left(#1\right)}
\renewcommand*{\ker}[1]{\operatorname{ker}\left(#1\right)}
